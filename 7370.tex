\documentclass[oneside]{article}

\usepackage[
  paperheight = 9in,
  paperwidth  = 6in,
  textheight  = 7.5in,
  textwidth   = 5in,
]{geometry}

\usepackage{
  amsmath,
  amssymb,
  amsthm,
  anyfontsize, % arbitrary font size
  bm,          % boldsymbol
  bookmark,    % replaces (and loads) hyperref
  fixltx2e,    % textsubscript
  mathrsfs,    % mathscr
  mathtools,   % allows prescript
  microtype,   % improved looks
  tikz-cd,     % better commutative diagrams
  thmtools,    % custom end to example environment
  7370,        % custom style file
}
\usetikzlibrary{matrix,decorations.pathreplacing}

\usepackage[
  hyperref = true,      % links to online documents
  backend  = bibtex,    % use bibtex instead of biber
  sorting  = nyt,       % sorts by (name, year, title)
  style    = alphabetic % citations look like [Har77]
]{biblatex}
\addbibresource{7370.bib}

\hypersetup{
  colorlinks = true,
  linkcolor  = blue,
  urlcolor   = cyan
}

\pagestyle{myheadings}
\markright{\hfill Ranks of elliptic curves\hfill}


\title{Ranks of elliptic curves}
\author{Daniel Miller and Ravi Ramakrishna}
\date{Fall 2014}


\begin{document}

\maketitle
\tableofcontents





% !TEX root = 7370.tex

\section{Introduction}





\subsection{Disclaimer}

These notes originated in the course MATH 7370: Ranks of elliptic curves, 
taught by Ravi Ramakrishna at Cornell University. The notes are not necessarily 
an exact reflection of the material as it was covered in class. 





\subsection{Motivation}

Our main references will be the preprints 
\cite{
  bhargava-shankar-1,
  bhargava-shankar-2,
  bhargava-shankar-3,
  bhargava-shankar-4,
  bhargava-skinner,
  bhargava-skinner-zhang}. 

The fundamental question is how to solve general diophantine equations 
\[
  f_1(x_1,\dots,x_m) = \cdots = f_n(x_1,\dots,x_m) = 0
\]
for the $f_i$ polynomials over $\dQ$ or $\dZ$ and the $x_i$ in $\dQ$ or $\dZ$. 
This is very hard, so we'll focus on curves. This comes down to solving 
equations $f(x,y)=0$ where $f\in \dZ[x,y]$ or $\dQ[x,y]$. Or we could 
homogenize and consider points on the associated projective curve. 

\begin{example}
Consider the equation $x^2+y^2=1$. There is one obvious solution $(0,1)$. From 
this we get all the other solutions by looking at lines $y=m x+1$ with 
$m\in \dQ$. The equation $x^2+(m x+1)^2=1$ already has one rational solution, 
so the other must be rational, and all rational solutions to $x^2+y^2=1$ are of 
this form. So solutions to $x^2+y^2=1$ are in bijection with $\dQ$. More 
geometrically, any nice (that is, smooth, proper and geometrically integral) 
curve $C_{/\dQ}$ that has one rational point is isomorphic (over $\dQ$) to 
$\dP^1_{/\dQ}$. 
\end{example}

Any smooth projective curve $C_{/\dQ}$ is the homogeneous zero-set of an 
equation $f(x,y,z)=0$. The set $C(\dC)$ is naturally a compact Riemann surface 
of some genus $g$. 

\begin{theorem}[Faltings]
If $C_{/\dQ}$ be a curve of genus $g>1$. Then $\# C(\dQ)<\infty$. 
\end{theorem}
There still isn't an effective algorithm for curves of genus $\geqslant 2$ that 
will produce the set of zeros of $f$. 

We've seen that when $g=0$, either $C(\dQ)=\varnothing$ or 
$C\simeq \dP^1$. When $g\geqslant 2$, Faltings' theorem tells us that 
$\# C(\dQ)<\infty$. Our concern is the remaining case $g=1$. If $C_{/\dQ}$ is 
a genus one curve and $C(\dQ)=\varnothing$, there isn't much to do. For the 
remainder, we will concentrate on nice curves $E_{/\dQ}$ together with a chosen 
point $0\in E(\dQ)$. Such curves are called \emph{elliptic curves}. 

Let $E_{/\dQ}$ be an elliptic curve. Basic algebraic geometry involving little 
more than Riemann-Roch shows that $E$ can be written in the form 
$y^2=x^3+A x+B$ with $A,B\in \dZ$. Projectively, this is
$y^2 z=x^3 + A x z^2+B z^3$. The point $(0:1:0)$ is the $0\in E(\dQ)$. 

\begin{example}
Consider the curve $y^2=x^3-9 x+9$. There are obvious solutions 
$(1,1)$, $(3,3)$. The line through them is $y=x$. Since the cubic 
$x^2=x^3-9 x-9$ already has two rational solutions, the third must be rational. 
This third point is  $(-3,-3)$. In general, if $P,Q,R\in E$, we say that 
$P+Q+R=0$ if $P,Q,R$ are the intersection points of a line and $E$ in 
$\dP^2$. For example, one can check that 
\begin{align*}
  (-3,3)+(1,1) &= (\frac 9 4, -\frac 3 8) \\
  (-3,3)+\left(\frac 9 4, -\frac 3 8\right) &= \left(\frac{57}{49},-\frac{111}{343}\right).
\end{align*} 
\end{example}

Any elliptic curve $E_{/\dQ}$ has the canonical structure of a group variety 
over $\dQ$. For our purposes, this means that there are regular maps 
$m:E\times E\to E$ and $i:E\to E$ that make $E(\dC)$ with the maps induced by 
$m,i$ an honest group with identity element $0$. But for any field 
$F\supset \dQ$, there is an abelian group $E(F)$ which is functorial in $F$. 

\begin{theorem}[Mordell-Weil]
Let $E_{/\dQ}$ be an elliptic curve. Then $E(\dQ)$ is finitely generated. 
\end{theorem}

Thus it makes sense to define the \emph{rank} of $E$ to be 
$\rank(E)=\rank_\dZ E(\dQ)$. Given a specific elliptic curve $E$, we might ask: 
what are its rank and torsion? The second part is easy. In his paper 
\cite{mazur-1977}, Mazur proved that the torsion part of $E(\dQ)$ is one of the 
following groups:
\begin{center}
\begin{tabular}{rl}
$\dZ/m$ & for $m\leqslant 10$ or $m=12$ \\
$(\dZ/2)\oplus (\dZ/2 n)$ & for $n\leqslant 4$ 
\end{tabular}
\end{center}
This paper is very beautiful, and inspired a lot of amazing mathematics. If $E$ 
is a specific elliptic curve, it is easy to explicitly check which of these the 
torsion subgroup of $E(\dQ)$ is. 





\subsection{\texorpdfstring{$L$}{L}-functions of elliptic curves}

For $p$ sufficiently large, $E_{/\dF_p}$ is an elliptic curve. Write 
$\# E(\dF_p)=p+1-a_p$; we have the \emph{Hasse bound} $|a_p|\leqslant 2\sqrt p$. 

We define the $L$-function of $E$ as 
\[
  L(E) = \prod_{p \text{ bad}} ? \times \prod_{p\text{ good}} \left(1-\frac{a_p}{p^s} + \frac{1}{p^{2s-1}}\right)^{-1} .
\]
There only finitely many bad primes (they are the ones dividing 
the discriminant $\Delta=-(4 A^3+27 B^2)$ if $E$ is the curve 
$y^2=x^3+A x+B$.), and the factors at the bad primes are rational functions in 
$p^s$. Nonetheless, this is the world's most awful definition! Let's give a 
better one. We know that 
$E[l^n]\simeq (\dZ/l^n)^2$. This group admits an action of 
$G_\dQ=\galois(\bar\dQ/\dQ)$, an uncountable group with a natural 
compact, totally disconnected topology. We can paste these actions together 
to get a representation $\rho:G_\dQ\to \GL_2(\dZ_l)$. There are conjugacy 
classes $\frobenius_p\in G_\dQ$, and $p\nmid \Delta$, we have 
\[
  \rho(\frobenius_p) \sim \begin{pmatrix} \alpha_p \\ & \beta_p \end{pmatrix} .
\]
It turns out that 
\[
  1-\frac{a_p}{p^s} + \frac{1}{p^{2s-1}} = \left(1-\frac{\alpha_p}{p^s}\right)\left(1 - \frac{\beta_p}{p^s}\right) .
\]
The function $L(E,s)$ may not seem that complicated, but note that the zeta 
function built from the trivial representation $G_\dQ\to \GL_1(\dC)$ is the 
Riemann zeta function 
\[
  \zeta(s)=\prod_p \left(1-\frac{1}{p^s}\right)^{-1} .
\]
So we should expect the behavior of $L(E,s)$ to be very subtle! Most zeta 
functions encountered in number theory come from Galois representations in this 
manner. 

\begin{conjecture}[Birch, Swinnerton-Dyer]
Let $E_{/\dQ}$ be an elliptic curve. Then $\rank(E)=\order_{s=1} L(E,s)$. 
\end{conjecture}

In fact, Birch and Swinnerton-Dyer predicted the leading term at $s=1$ of 
$L(E,s)$ in terms of arithmetic data attached to $E$. The conjecture implicitly 
includes the assertion that $L(E,s)$ has a meromorphic continuation past $s=1$. 
By the work of Wiles and his followers, we know that $E$ is modular, hence its 
$L$-function agrees with that of a modular form. Hecke showed that modular 
$L$-functions have such meromorphic continuations, so no problems there. 

It is known that $L(E,s)$ satisfies a functional equation 
\[
  L(E,s) = \pm ? L(E,2-s) 
\]
where $?$ is easily computable. If the sign is negative, we know that 
$L(E,1)=0$, which suggests that $\rank(E)>0$.

\begin{conjecture}[Goldfeld, Katz-Sarnak]
Let $E_{/\dQ}$ be an elliptic curve. Then 
\[
  \rank(E) = \begin{cases} 0 & \text{50\% of the time} \\ 1 & \text{50\% of the time.} \end{cases}
\]
\end{conjecture}

We have to be careful about what we mean by ``50\% of the time'' as there are 
infinitely many elliptic curves over $\dQ$. To make probabilistic statements 
precise, we'll order elliptic curves. 





\subsection{Asymptotics of rank}

Given $A,B\in \dZ$ with $4 A^3+27 B^2\ne 0$, write $E_{A,B}$ for the elliptic 
curve $y^2=x^3+A x+B$. Define its \emph{height} to be 
$\height(E)=\max(4 |A|^3,27 |B|^2)$. It's easy to check that asymptotically, 
about $X^{5/6}$ curves have height $\leqslant X$. Let $A(X)$ be the average 
rank of the set of elliptic curves with height $\leqslant X$. Assuming the 
Generalized Riemann  Hypothesis, these were the bounds on $A(X)$ prior to 
Bhargava's work: 
\begin{align*}
  \limsup_{X\to \infty}  A(X) \leqslant 
    & 2.3 && \text{\cite{brumer-1992}} \\
    & 2.0 && \text{\cite{heath-brown-2004}} \\
    & 1.79 && \text{\cite{young-2006}}
\end{align*}
Unconditionally, all that was known was the trivial bounds
\[
  0\leqslant \liminf_{X\to \infty} A(X) \leqslant \limsup_{X\to \infty} A(X) \leqslant \infty .
\]
Now, after Bhargava's work, we have
\[
  0.2 \leqslant \liminf_{X\to \infty} A(X) \leqslant \limsup_{X\to \infty} A(X) \leqslant 0.885 .
\]
We also know that $\order_{s=1} L(E,s) = \rank E(\dQ)$ at least 
$66.43\%$ of the time, and there is a strategy to get $100\%$. Unfortunately, 
this strategy only says things about the (conjecturally $100\%$ of) elliptic 
curves with $\rank\leqslant 1$. 

There is a folklore question: is $\rank(E)$ bounded? About 15 years ago, most 
people thought this wasn't bounded, now, most people think it is bounded. 





\subsection{Proof strategy}

For any abelian group $A$, we it is trivial that
$\rank(A)\leqslant \dim_{\dF_p}(A/p)$. We can prove that the set of elliptic 
curves with $\dZ/p\subset E[p]$ is density zero, so it won't affect density 
arguments. 

There is an exact sequence 
\[\begin{tikzcd}
  0 \ar[r] 
    & E[p](\overline\dQ) \ar[r] 
    & E(\bar\dQ) \ar[r, "p"]
    & E(\bar\dQ) \ar[r] 
    & 0 .
\end{tikzcd}\]
Take $G_\dQ$-invariants and pass to the long exact sequence in Galois 
cohomology: 
\[\begin{tikzcd}
  0 \ar[r] 
    & E(\dQ)[p] \ar[r] 
    & E(\dQ) \ar[r, "p"] 
    & E(\dQ) \ar[r] 
    & \h^1(G_\dQ,E[p]) \ar[r] 
    & \cdots
\end{tikzcd}\]
Some fiddling around involving completions of $\dQ$ gives a short exact sequence 
\[\begin{tikzcd}
  0 \ar[r] 
    & E(\dQ)/p \ar[r] 
    & \selmer_p(E) \ar[r] 
    & \sha(E)[p] \ar[r]
    & 0 .
\end{tikzcd}\]
In the 60s, Cassels discovered a nice geometric way of describing $\selmer_p E$. 
Elements correspond to some geometric objects (a map $C\to E$), which in turn 
correspond to a embedding $C\hookrightarrow \dP^{p-1}$ (at least if 
$p\geqslant 3$). When $p=3$, $C$ is the zero locus of a ternary cubic form. 

Ternary cubics have two invariants $I,J$. These correspond to $A$ and $B$. So 
rather than counting elliptic curves, we can count ternary cubics (up to 
equivalence). 

The quadratic forms $x^2+y^2$ and $(u+v)^2+v^2$ are isomorphic over $\dZ$. 
General quadratic forms $a x^2+b x y+c y^2$ over $\dZ$ have an invariant 
$b^2-4 a c$ (invariant for action of $\SL_2(\dZ)$). Essentially, you have 
to find a fundamental domain for the action of $\SL_2(\dZ)$ on the upper 
half plane. It's a lot harder to find a fundamental domain for the action 
of $\SL_3(\dZ)$ on a bigger space. 

For $p\leqslant 5$, you count equivalences of forms; this is done by 
finding a fundamental domain for the action of some arithmetic group. 
This has been done for $p=2,3,5$. 

Essentially, one counts lattice points in $G\backslash V_\dZ$. The problem 
is, the fundamental domains can have cusps. It turns out that the 
nonzero elements of $\selmer_p E$ appear in the ``main body'' of the 
fundamental domain, and only $0$ appears in the cusps. 

\begin{theorem}[Bhargava-Shankar]
The average size of $\selmer_p E=p+1$ for $p=2,3,5$. 
\end{theorem}

Put $x=\dim_{\dF_p} E(\dQ)/p$. It is easy to check that 
$(p^2-p) x+2 p-p^2 \leqslant p^x$. Note that $p^x$ is $p+1$ on average. 
It follows that on average, $\rank E\leqslant 1+\frac{1}{p(p-1)}$. 
If we could prove this for all $p$, we get average $\leqslant 1$. Bhargava 
and Shankar use work of the Dokchitser's to get bounds $<1$ without proving 
that the average of $\#\selmer_p E=p+1$ for all $p$. 





\subsection{Review of the overview}

\emph{Note}: the above introduction was given in a department-wide lecture. 
What follows is a general overview given in the first day the class met. 

Let $E_{/\dQ}$ be an elliptic curve. This is a smooth genus $1$ curve over 
$\dQ$ of the form $y^2=x^3+A x+B$, where $A,B\in \dZ$, such that 
\[
  p^4\mid A\Rightarrow p^6\nmid B .
\]
It is a basic fact that every isomorphism class of elliptic curves over 
$\dQ$ has a unique representative of this form. 

Throughout, we'll write an equation $y^2=x^3+A x+B$ to mean the subvariety of 
$\dP^2$ cut out by the homogenization $y^2 z=x^3+A x z^2+B z^3$ of this 
equation. So $E(\dQ)$ consists of solutions to $y^2=x^3+A x+B$, as well as the 
point $(0:1:0)$ ``at infinity.'' Some useful facts:
\begin{enumerate}
\item
$E(\dQ)$ is a finitely-generated abelian group (Mordell-Weil). 

\item
$E(\dQ)_\mathrm{tors}$ is ``understood'' All possible such 
subgroups have been written down (see Mazur's list above), and they all 
occur. There is an effective algorithm to determine the torsion part of the 
Mordell-Weil group of an elliptic curve $E$. Since the group law is given by 
polynomials, the ``multiply by $n$'' map $[n]:E\to E$ is a polynomial. 
Simply check whether the roots of $[n]$ lie in $\dQ$. There are only finitely 
many possible $n$, so the algorithm will terminate. 

\item
$\#E(\dQ)_\mathrm{tors}=1$ one hundred percent of the time. 
\end{enumerate}

Recall that the \emph{height} of $E_{A,B}:y^2 z=x^3+A x z^2+b z^3$ is 
$\max(4|A|^3,27 B^2)$. This definition is actually 
pretty natural if you know the definition of the discriminant of $E$. 

Recall that our goal is to study the average rank of $E_{/\dQ}$ with height 
$\leqslant X$, as $X\to \infty$. That is, we are interested in the asymptotics 
of 
\[
  \lim_{X\to \infty} \frac{\sum_{\height E\leqslant X} \rank(E)}{\#\{E:\height E\leqslant X\}} .
\]
Currently it is not known if this limit exists, but conjecturally it is $1/2$. 
Note that $\#\{E:\height E\leqslant X\}=O(X^{1/3}\cdot X^{1/2})=O(X^{5/6})$. We 
have to be careful, because some of the curves $E_{A,B}$ will be singular, but 
this happens on a thin set, so it will not affect our calculations. 

Note that for any finitely-generated abelian group $A$, we have 
\[
  \rank(A) = \dim_\dQ(A\otimes\dQ) \leqslant \dim_{\dF_p}(A\otimes\dF_p) .
\]
So to get upper bounds on $\rank(E)$, it suffices to bound 
$\dim_{\dF_p}(E(\dQ)/p)$. 

\begin{theorem}
Let $E_{/\dQ}$ be an elliptic curve. Then $E(\overline\dQ)$ is a divisible 
group. 
\end{theorem}
\begin{proof}
Given $n\in \dZ$ and $x\in E(\overline\dQ)$, we need some $y\in E(\overline\dQ)$ 
such that $n\cdot y=x$. The coefficients of the polynomial $[n]$ will be 
algebraic over $\dQ$, so all solutions to $[n] y=x$ will be algebraic over $\dQ$. 
\end{proof}

By \cite[Cor 5.11]{moonenABV}, we have $E[n](\overline\dQ)\simeq (\dZ/n)^2$ for 
all $n\geqslant 1$. Thus $E(\overline\dQ)_\mathrm{tors}\simeq (\dQ/\dZ)^2$. 
Since $E(\overline\dQ)$ is divisible, we have an exact sequence 
\begin{equation}\label{eq:ell-ses}
\begin{tikzcd}
  0 \ar[r] 
    & E(\overline\dQ)[p] \ar[r] 
    & E(\overline\dQ) \ar[r, "p"] 
    & E(\overline\dQ) \ar[r] 
    & 0 .
\end{tikzcd}
\end{equation}
Recall that $E(\dC)\simeq S^1\times S^1$ as real Lie groups. The finite group 
$E(\overline\dQ)[p]\simeq (\dZ/p)^2$ comes with a standard pairing, the 
\emph{Weil pairing}, coming from the cup product on $\h^1(\dQ,E[p])$. 

Let $\overline\dQ$ be an algebraic closure of $\dQ$, and let 
$G_\dQ=\galois(\overline\dQ/\dQ)$. This is an uncountable, totally disconnected 
compact topological group, and its cohomology has a good duality theory; see for 
example \cite{neukirch-schmidt-winberg-2008}. There is an analogy between 
$G_\dQ$ and 3-manifolds, see e.g.~\cite{morishita-2012}. 

The group $G_\dQ$ acts on $E(\overline\dQ)$, respecting the group structure. So 
we can take its cohomology. 
The functor $M\mapsto M^{G_\dQ}$ is left exact, so we can do the usual nonsense 
with enough injectives and derived functors to get $\h^\bullet(G_\dQ,M)$. Or we 
can write a direct definition using cocycles and coboundaries. Often, to save 
space, if $F$ is a field, $G_F$ its absolute Galois group and $M$ a 
$G_F$-module, we'll write $\h^\bullet(F,M)$ instead of $\h^\bullet(G_F,M)$. 

A commutative diagram 
\[\begin{tikzcd}
  \overline\dQ \ar[r, hook] 
    & \dC \\
  \dQ \ar[r, hook] \ar[u, hook] 
    & \dR \ar[u, hook] 
\end{tikzcd}\]
gives rise to an injection $G_\dR=\galois(\dC/\dR)\monic G_\dQ$. Also, for 
each prime $l$ we get an inclusion $G_{\dQ_l}\monic G_\dQ$. The groups 
$G_{\dQ_l}$ are much bigger than $G_\dR=\dZ/2$, but they're pretty well-behaved 
[for example, they're pro-solvable in a nice way]. The group 
$G_\dQ$ is very poorly understood. We know its abelianization quite well, and 
2-dimensional representations reasonably well, but higher-dimensional 
representations not well at all. 

The long exact sequence in cohomology coming from \eqref{eq:ell-ses} gives 
rise to a natural short exact sequence 
\[\begin{tikzcd}
  0 \ar[r] 
    & E(\dQ)/p \ar[r] 
    & \h^1(G_\dQ,E[p]) \ar[r] 
    & \h^1(G_\dQ,E)[p] \ar[r] 
    & 0
\end{tikzcd}\]
Unfortunately, $\h^1(G_\dQ,E[p])$ is infinite-dimensional, so this doesn't seem 
very helpful. But we can extend this to a diagram 
\[\begin{tikzcd}
  0 \ar[r] 
    & E(\dQ)/p \ar[r] \ar[d] 
    & \h^1(G_\dQ,E[p]) \ar[r] \ar[d] 
    & \h^1(G_\dQ,E)[p] \ar[r] \ar[d] 
    & 0 \\
  0 \ar[r] 
    & \prod_v E(\dQ_v)/p \ar[r] 
    & \prod_v \h^1(G_{\dQ_v},E[p]) \ar[r] 
    & \prod_v \h^1(G_{\dQ_v},E)[p] \ar[r] 
    & 0 .
\end{tikzcd}\]
Here, by convention $v$ ranges over all primes \emph{and $\infty$}, and we 
put $\dQ_\infty=\dR$. The vertical maps come from basic functoriality 
(restriction) of group cohomology. Put 
\[
  \selmer_p(E) = \ker\left(\h^1(G_\dQ,E[p]) \to \prod_v \h^1(G_{\dQ_v},E)[p]\right) .
\]
This is finite-dimensional, and $E(\dQ)/p\monic \selmer_p E$. The group 
$\selmer_p(E)$ is measuring some local-global stuff. That is, cohomology 
classes in $\h^1(G_\dQ,E[p])$ that ``are'' trivial locally everywhere. 

In \cite{cassels-1962}, Cassels showed that elements of $\selmer_p(E)$ are in 
bijection with locally soluble $p$-coverings of $E$. 

Briefly, the map $p:E\to E$ has field of definition $\dQ$. Consider the varieties 
$U:x^2+y^2=1$ and $V:x^2+y^2=3$. There is an isomorphism 
$U_{\dQ(\sqrt 3)} \to V_{\dQ(\sqrt 3)}$ by $(x,y)\mapsto (\sqrt 3x,\sqrt 3 y)$, 
but this isomorphism isn't defined over $\dQ$. A \emph{locally soluble 
$p$-covering} of $E$ is an isomorphism $\phi:C\iso E$ (not necessarily 
defined over $\dQ$) such that $[p]\circ \phi$ is defined over $\dQ$. Moreover, 
$C$ must have points over $\dR$ and all $\dQ_l$. 

Locally soluble $p$-coverings of $E$ give degree-$p$ divisors on $C$. This 
gives a map $C\to \dP^{p-1}$, which is an embedding of degree $3$ if 
$p\geqslant 3$. 

Say $p=3$. A degree-$3$ curve in $\dP^2$ is easy to describe. So elements in 
$\selmer_3(E)$ correspond to degree-$3$ cubics over $\dQ$ (up to equivalence). 

Let's switch gears. Consider binary quadratic forms over $\dZ$. These are just 
polynomials $a x^2+b x y+c y^2$ for $a,b\in \dZ$. These have a natural 
$\SL_2(\dZ)$-action coming from 
\[
  \begin{pmatrix} x \\ y \end{pmatrix} \mapsto \begin{pmatrix} a & b \\ c & d \end{pmatrix} = \begin{pmatrix} a x+b y \\ c x+d y \end{pmatrix} .
\]
But $\SL_2(\dZ)$ has generators $\begin{pmatrix} 1 & 1 \\ & 1 \end{pmatrix}$ and 
$\begin{pmatrix} & -1 \\ 1 \end{pmatrix}$. 

Claim: $d=b^2-4 a c$ is an invariant of the action of $\SL_2(\dZ)$. We'll only 
check $\begin{pmatrix} 1 & 1 \\ & 1 \end{pmatrix}$-invariance. We get 
\begin{align*}
  a(x+y)^2+b (x+y)y+c y^2 
    &= a x^2 + (2 a+b)x y + (c+b+a) y^2 ,
\end{align*}
and simply check that $(2 a+b)^2 - 4 a(a+b+c)=b^2-4 a c$. 

\begin{theorem}[Gauss]
Given $d$, there exist only finitely many inequivalent binary quadratic forms 
over $\dZ$ with discriminant $d$. 
\end{theorem}

It's natural to ask: ``how many are there?'' We'll ask the exact same question 
for ternary forms. In that case there are two invariants $I,J$. Proving that these 
are invariant is a simple computation. What is harder is the analogue of 
\[
  \dZ[b^2-4 a c]= \dZ[a,b,c]^{\SL_2(\dZ)} .
\]

A general theorem of Borel and Harish-Chandra says that there 
are still only finitely many equivalence classes of forms with any given 
pair of invariants. 

Q. what values $d$ occur? [easy: $d\equiv 0$ or $1\pmod 4$]

Q. What is the arithmetic significance of 
\[
  h(d)=\#\{\text{inequivalent quadratic forms with discriminant }d\}
\]
It turns out that $h(d)$ is the \emph{narrow class number} of $\dQ(\sqrt d)$. 
This measures failure of unique factorization in that field. 

Q. What is the average value of $h(d)$?

\begin{theorem}[Mertens, Siegel]
(a) For $-X<d<0$, the average value of $h(d)$ is $\frac{\pi}{18} X^{3/2} + O(X^{3/2-\varepsilon})$ for some explicit $\varepsilon>0$. 

(b) For $0<d<X$, the average value of $h(d)\log(\epsilon_d)$ is 
$\frac{\pi^2}{18} X^{3/2}+O(X^{3/2-\varepsilon})$. Here 
$\epsilon_d$ is a fundamental unit of $\dQ(\sqrt d)$. 
\end{theorem}

What's different about $p=2$ and $p>2$? When $p=2$, 
$C\to \dP^{p-1}$ isn't an embedding. But it is a degree-2 cover with four 
branch points. This comes down to counting binary quartic forms up to 
equivalence. Let 
\[
  a x^4 + b x^3 y + c x^2 y^2 + d x y^3 + e y^4 
\]
be such a form. The group $\SL_2(\dZ)$ acts as before. If 
\begin{align*}
  I &= 12 a e - 3 b d + c^2 \\
  J &= 72 a c e + 9 b c d - 27 a d^2 - 27 e b^2 - 2 c^3 .
\end{align*}
then we have 

\begin{theorem}
$\dZ[I,J] = \dZ[a,b,c,d,e]^{\SL_2(\dZ)}$. 
\end{theorem}
\begin{proof}
Find source [not Hilbert!]
\end{proof}

So counting elements of $\selmer_2(E)$ of $E$ of height $X$ corresponds to counting 
binary quartic forms via $(I,J)\leftrightarrow (A,B)$. 

\begin{theorem}
The number of such forms up to height $X$ with 
\begin{align*}
  4\text{ real roots} && \frac{4}{\zeta(2)}{135} X^{5/6}+O(X^{3/4}) \\
  2\text{ real roots} && 32 \frac{\zeta(2)}{135} X^{5/6} + O(X^{3/4}) \\
  \text{no real roots} && 8 \frac{\zeta(2)}{135} X^{5/6} + O(X^{3/4}) .
\end{align*}
\end{theorem}

\begin{corollary}
When ordered by height, $\average(\#\selmer_2)=2+1$. 
\end{corollary}


% !TEX root = 7370.tex

\section{Background}

We'll thread a very narrow path to the main theorems, assuming a bunch of 
facts along the way. A good source on the algebraic geometry we'll do is
\cite[ch.2,3]{silverman-2009}. 





\subsection{Riemann-Roch}

We'll spend the rest of this subsection explaining the terms in the following 
theorem. 

\begin{theorem}[Riemann-Roch]
Let $C_{/k}$ be a smooth projective curve. Let $K$ be the canonical divisor and 
$D$ a divisor. Then $l(D)-l(K-D)=\deg D-g+1$. 
\end{theorem}

For us, a curve $C_{/k}$ is the zero locus of a (non-constant) polynomial 
$f\in k[x,y]$. We stress the fact that $k$ is \emph{not} necessarily 
algebraically closed here. Pictorially, smoothness means that $C$ has no 
cusps or self-intersections. For example, $f$ can't be something like 
$x y$ or $x^2-y^3$. More formally, we want the tangent space to be 
$1$-dimensional at every point on $C$. For any $c\in C$, we have a 
matrix 
$\begin{pmatrix} \frac{\partial f}{\partial x}(c),\frac{\partial f}{\partial y}(c)\end{pmatrix}$
which needs to be nondegenerate, i.e.~the partials of $f$ never 
simultaneously vanish. Note that here, ``$c\in C$'' means $c\in C(\bar k)$, so 
we consider points not defined over $k$. We'll constantly move between the 
affine curve $V(f)\subset \dA^2$ and the projective curve in $\dP^2$ cut out by 
the projectivisation of $f$. 

A \emph{divisor} on a curve $C$ is a formal $\dZ$-linear combination of points 
(defined over $\bar k$). 

\begin{example}
Let $C:x^2+y^2=10$. Let $P=(4,i\sqrt 6)$ and $Q=(4,-i\sqrt 6)$. Then $P$ and 
$Q$ are not defined over $\dQ$, but $P+Q$ is a divisor on $C$ defined over 
$\dQ$. 
\end{example}

The absolute Galois group $G_k$ of $k$ acts on the space of divisors on $C$. We 
say a divisor $D$ is \emph{defined over $k$} if $D=D^\sigma$ for all 
$\sigma\in G_k$. 

We put an equivalence relation on divisors: $D_1\sim D_2$ if 
$D_1-D_2=\divisor(f)$ for some rational function $f$ on $C$. 
So we need to define $\divisor(f)$. Let $k[C]=k[x,y]/f$ be the ring of regular 
functions on $f$, and let $k(C)$ be the field of fractions of $k[C]$; we call 
$k(C)$ the field of \emph{rational functions} on $C$. Any rational function 
$f$ induces a morphism $f:C\to \dP^1$. Since $C$ is projective, this map is 
either constant or surjective. We can now put 
\[
  \divisor(f) = \sum_{c\in C(\bar k)} \order_c(f) \cdot c.
\]
We put $\picard(C)=\divisors(C)/\divisor(k(C)^\times)$. 

Now we need to define the canonical divisor. It is the class of a nonvanishing 
differential on $C$. More formally, it's the divisor class corresponding to the 
line bundle $\Omega_C^1$. 

We put $l(D)=\dim \sL(D)$, where 
\[
  \sL(D) = \{f:C\epic \dP^1:\divisors(f)+D\geqslant 0\} \cup \{0\}.
\]
It is a theorem that $l(D) < \infty$ for all divisors $D$. 

Aside: let $[L:\dQ]<\infty$, i.e.~let $L$ be a number field. Then there is an 
exact sequence 
\[\begin{tikzcd}
  0 \ar[r] 
    & \cO_L^\times \ar[r] 
    & L^\times \ar[r] 
    & I_L \ar[r] 
    & \picard(O_L) \ar[r] 
    & 0 ,
\end{tikzcd}\]
where $I_L$ is the group of fractional ideals in $L$. See 
\cite{lorenzini-1996} for more on the parallel between 
algebraic number theory and algebraic geometry. This is analogous to the 
sequence 
\[\begin{tikzcd}
  0 \ar[r] 
    & k^\times \ar[r] 
    & k(C)^\times \ar[r, "\divisor"] 
    & \divisors^\circ(C) \ar[r] 
    & \picard^\circ(C) \ar[r] 
    & 0 .
\end{tikzcd}\]
Here the degree $\deg(\sum n_c c) = \sum n_c$. It is a basic fact that for 
$f\in k(C)^\times$, we have $\deg(f)=0$. In fact, $\picard(X)$ makes sense 
for $X$ any scheme (e.g.~a curve or the spectrum of $\cO_L$). 

If $C_{/\dQ}$ is a smooth projective curve, then $C(\dC)$ will be a compact 
Riemann surface, hence a torus with $g$ holes. We call $g$ the \emph{genus} of 
$C$. Geometrically, $g=\dim \h^0(\Omega^1)$. 

\begin{theorem}\label{thm:basic-facts}
\begin{enumerate}
\item $\deg D<0\Rightarrow l(D)=0$. 
\item $l(D)<\infty$. 
\item $D\sim D'\Rightarrow \sL(D)\simeq \sL(D')$. 
\item $l(K_C)=g$ and $\deg(K_C)=2 g-2$. 
\item $\deg(D)>2 g-2\Rightarrow l(D)=\deg(D)-g+1$. 
\end{enumerate}
\end{theorem}
\begin{proof}Choose $D$ cleverly (i.e.~$0$ or $K_C$) in the Riemann-Roch 
Theorem, and use $l(0)=1$. 
\end{proof}





\subsection{Elliptic curves}

If we're analyzing the case $g=0$, diophantine properties were known to the 
Greeks. For $g\geqslant 2$, Faltings tells that $\#C(\dQ)<\infty$. The 
remaining case is $g=1$. 

\begin{definition}
Let $k$ be a field. An \emph{elliptic curve over $k$} is a smooth projective 
curve $E_{/k}$ together with a point $0\in E(k)$. 
\end{definition}

Let $E_{/k}$ be an elliptic curve, $e\in E(k)$. What is $l(e)$? From 
\autoref{thm:basic-facts}, we get: 
\begin{center}
\begin{tabular}{c|cl}
$n$ & $l(n\cdot e)$ & basic of $\sL(n\cdot e)$ \\ \hline
1 & 1 & $\{1\}$ \\
2 & 2 & $\{1,x\}$ \\
3 & 3 & $\{1,x,y\}$ \\
4 & 4 & $\{1,x,y,x^2\}$ \\
5 & 5 & $\{1,x,t,x^2,x y\}$ \\
6 & 6 & $\{1,x,y,x^2,x y, x^3\text{ or } y^2\}$ 
\end{tabular}
\end{center}
The last set could have seven elements, but $\sL(6\cdot e)$ is 
$6$-dimensional. It follows that there is a dependence relation 
\[
  \alpha_0 + \alpha_1 x + \alpha_2 y+ \alpha_3 x^2 + \alpha_4 x y+\alpha_5 x^3 + \alpha_6 y^2 = 0 .
\]
We claim that $(\alpha_4,\alpha_6)\ne (0,0)$. Basic consideration of the 
orders of poles in the remaining equation at $e$ shows this. We can assume that 
$\alpha_6\ne 0$. Say $\alpha_6=0$ (then $\alpha_4\ne 0$). Consider the change 
of variables 
\begin{align*}
  x &\mapsto x + cy \\
  y &\mapsto y. 
\end{align*}
The new coefficient of $y^2$ is $\alpha_3 c^2 + \alpha_4 c$, and we can choose 
$c$ so that this is $\ne 0$. There's a better proof that $\alpha_6\ne 0$. If 
$\alpha_6=0$, since $\{1,x,y,x^2,x y,x^3\}$ is a basis of $\sL(6\cdot e)$, all 
the $\alpha_i=0$, which we can assume is not the case. 

Divide through by $\alpha_6$ to get the following:
\[
  y^2 + y (\alpha_2+\alpha_4 x) + \left(\frac{\alpha_2+\alpha_4 x}{2}\right)^2 - \left(\frac{\alpha_2+\alpha_4 x}{2}\right)^2 + \alpha_3 x^3 + \alpha_1 x + \alpha_0 = 0 .
\]
Replace $y$ by $y+\frac{\alpha_2+\alpha_4 x}{2}$; we get: 
\[
  y^2 = \gamma_3 x^3 + \gamma_2 x^2 + \gamma_1 x + \gamma_2 .
\]
Replace $y$ by $\gamma_3^2 y$ and $x$ by $\gamma_3 x$; this gives 
$x^3$ and $y^2$ the same coefficient. Rescale and we get 
\[
  y^2 = x^3 + \delta_2 x^2 + \delta_1 x + \delta_0 .
\]
Replace $x$ by $x-\frac{\delta_2}{3}$, and we get an equation 
\[
  y^2 = x^3 + A x+B 
\]
with $A,B\in k$. 

Note we have used that $k$ has characteristic not $2$ or 
$3$. Completion of squares doesn't work in characteristic $2$, and the 
first step in solving cubics doesn't work in characteristic $3$. So in 
characteristic $3$, you can get $y^2=(\text{cubic})$, but for characteristic 
$2$, you can't really simplify the equation at all. In general, if $A$ is a 
$d$-dimensional abelian variety, it is noted in \cite{serre-tate-1968} that 
primes $p\leqslant 2 d+1$ can be especially nasty. 

We will need to reduce elliptic curves modulo $2$ and $3$. 
Given $E_{/\dQ}$, there is an integer $N$, called the \emph{conductor} of $E$, 
that measures the ``badness'' of the singularities in the reductions of $E$. 
For $p\geqslant 3$, we have $v_p(N)\leqslant 2$. For 
$p=3$, we just have $v_3(N)\leqslant 4$, and for $p=2$, we can have 
$v_2(N)=6$. 

We'll continue under the assumption that $k=\dQ$. Write our curve 
$E_{/\dQ}$ as 
\[
  y^2=x^3 + \frac{N_1}{D} x + \frac{N_2}{D} ,
\]
with $N_1,N_2,D\in \dZ$. Make the change of variables 
\begin{align*}
  x &\mapsto x/D^2 \\ 
  y &\mapsto y/D^3 .
\end{align*}
Then multiply through by $D^6$ and relabel. We get 
\[
  y^2=x^3+ A x+ B x ,
\]
with $A,B\in \dZ$. For any prime $p$, say 
$p^4\mid A$ and $p^6\mid B$. Make the change of variables
\begin{align*}
  x &\mapsto p^2 x \\ 
  y &\mapsto p^3 y .
\end{align*}
you can rescale again until one of these possibilities fail. We end up with 
a Weierstrass form $y^2=x^3+A x+B$ with $p^4\mid A\Rightarrow p^6\nmid B$. This 
is the \emph{minimal model} of $E$. 

Now we can study $E_{/\dQ}$ and write $E=E_{A,B}$, given by $y^2=x^3+A x+B$, 
for $A,B\in \dZ$ with $(p^4\nmid A\text{ or }p^6\nmid B)$. Recall that 
$E_{A,B}$ is singular exactly if 
\[
  \begin{pmatrix} 2 y & 3 x^2+A\end{pmatrix} = 0
\]
for some $(x,y)\in E(\overline\dQ)$. This can happen only if 
$27 B^2+4 A^3=0$. The set of such $(A,B)$ is thin. 

This procedure works over $\dZ[\frac 1 6]\subset \dZ$. 


\section{Main stuff}

Let $k$ be a number field, $C_{/k}$ a smooth proper curve. Let $g$ be the 
genus of $C$. If $g=0$, either $C(k)=\varnothing$, or $C\simeq \dP^1_{/k}$. 
If $g>1$, then Faltings gave an ineffective proof that $\# C(k)<\infty$. 
If $g=1$, then either $C(k)=\varnothing$, or, if $C(k)\ne\varnothing$, then 
$C(k)$ is a finitely generated abelian group. Recall that the Birch and 
Swinnerton-Dyer conjecture is that $\rank(E(k))=\order_{s=1} L(E,s)$. The 
function $L(E,s)$ satisfies a functional equation $L(E,s) \sim \pm L(E,2-s)$. 
We expect $(-1)^{\rank(E)}=\pm$ accordingly. This is the ``sign conjecture.'' 
Kolyvagin and others have proved that if $L(E,s)$ has a pole of order at most 
$1$ at $1$, then BSD holds for $E$. A folklore question is: is 
$\sup_{E/\dQ} \rank(E)<\infty$?

Any $E_{/\dQ}$ can be written as $y^2=x^3+A x+B$ where $A,B\in \dZ$, and 
$p^6\mid B\Rightarrow p^4\nmid A$. The \emph{height} of $E$ is 
$H(E)=\max(4|A|^3,27 B^2)$. This is asymptotically degree-six in the roots 
of $x^3+A x+B$. We could have ordered $E$ by (Galois-theoretic) conductor. The 
reason we use height is that the number of elliptic curves with height 
$\leqslant X$ is $C X^{5/6}+(\text{error term})$ for some constant $C$. On the 
other hand, we have no idea how many elliptic curves there are with bounded 
conductor. So even though height isn't as conceptually elegant, we'll order 
elliptic curves by it. 

Given $E_{/\dQ}$ with algebraic rank $3$, we could numerically show that the 
analytic rank is $\leqslant 3$, then use sign considerations to get t
$r_\mathrm{an}\in \{1,3\}$. If $r_\mathrm{an}=1$, we know that 
$r_\mathrm{alg}=1$, a contradiction, hence $r_\mathrm{an}=3$. Thus, for 
(specific) elliptic curves with algebraic rank $3$, it is possible to prove 
BSD. At the present, no one knows how to do this for a family of elliptic 
curves. From the Kummer exact sequence $0\to E[p] \to E\xrightarrow p E \to 0$, 
one gets a short exact sequence
\[\begin{tikzcd}
  0 \ar[r] 
    & E(\dQ)/p \ar[r] 
    & \h^1(\dQ,E[p]) \ar[r] 
    & \h^1(\dQ,E)[p] 
    & 0 .
\end{tikzcd}\]
One has a similar short exact sequence for each place $v$. This yields a 
(now-familiar) commutative diagram:
\[\begin{tikzcd}
  0 \ar[r] 
    & E(\dQ)/p \ar[r] \ar[d] 
    & \h^1(\dQ,E[p]) \ar[r] \ar[d] 
    & \h^1(\dQ,E)[p] \ar[d] \ar[r] 
    & 0 \\
  0 \ar[r] 
    & \prod E(\dQ_v)/p \ar[r] 
    & \prod \h^1(\dQ_v,E[p]) \ar[r] 
    & \prod \h^1(\dQ_v,E)[p] \ar[r] 
    & 0.
\end{tikzcd}\]
Even better, we can write 
\[
  \selmer_p(E)  \ker\left(\h^1(\dQ,E[p])\to \h^1(\dA,E)[p]\right) ,
\]
where $\dA$ is the ring of adeles over $\dQ$ and ``$\h^1$'' denotes \'etale 
cohomology. 

We'd like to give a geometric interpretation of $\h^1(\dQ,E)$ and 
$\h^1(\dQ,E[p])$. First, note that 
$\dim (E(\dQ)/p)=\rank(E)+\delta$, where $\delta=\dim E[p](\dQ)\leqslant 2$. 
With respect to height, $\delta>0$ on a density zero set, so we may as well 
assume $\delta=0$. Recall there is a short exact sequence 
\[\begin{tikzcd}
  0 \ar[r] 
    & E(\dQ)/p \ar[r] 
    & \selmer_p(E) \ar[r] 
    & \sha[p] \ar[r] 
    & 0 .
\end{tikzcd}\]
Separating $\selmer_p(E)$ into its contributions from $E(\dQ)/p$ and 
$\sha[p]$ is very difficult. One conjectures that 
$\sha=\ker(\h^1(\dQ,E) \to \h^1(\dA,E))$ is finite for all $E$. This conjecture 
implies that $\sha[p]=0$ for almost all $p$. Cassels has constructed a perfect 
pairing $\sha\times \sha\to \dQ/\dZ$. So if $\#\sha<\infty$, then $\#\sha$ 
should be a perfect square. The ``$\sha$-part'' of BSD can be computed in 
some examples, and it always turns out to be a perfect square. 

[\ldots recap of strategy for Bhargava's proof\ldots]





\newpage
\printbibliography[heading=bibintoc]
\end{document}
