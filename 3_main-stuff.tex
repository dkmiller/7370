% !TEX root = 7370.tex





\section{Main stuff}

Let $k$ be a number field, $C_{/k}$ a smooth proper curve. Let $g$ be the 
genus of $C$. If $g=0$, either $C(k)=\varnothing$, or $C\simeq \dP^1_{/k}$. 
If $g>1$, then Faltings gave an ineffective proof that $\# C(k)<\infty$. 
If $g=1$, then either $C(k)=\varnothing$, or, if $C(k)\ne\varnothing$, then 
$C(k)$ is a finitely generated abelian group. Recall that the Birch and 
Swinnerton-Dyer conjecture is that $\rank(E(k))=\order_{s=1} L(E,s)$. The 
function $L(E,s)$ satisfies a functional equation $L(E,s) \sim \pm L(E,2-s)$. 
We expect $(-1)^{\rank(E)}=\pm$ accordingly. This is the ``sign conjecture.'' 
Kolyvagin and others have proved that if $L(E,s)$ has a pole of order at most 
$1$ at $1$, then BSD holds for $E$. A folklore question is: is 
$\sup_{E/\dQ} \rank(E)<\infty$?

Any $E_{/\dQ}$ can be written as $y^2=x^3+A x+B$ where $A,B\in \dZ$, and 
$p^6\mid B\Rightarrow p^4\nmid A$. The \emph{height} of $E$ is 
$H(E)=\max(4|A|^3,27 B^2)$. This is asymptotically degree-six in the roots 
of $x^3+A x+B$. We could have ordered $E$ by (Galois-theoretic) conductor. The 
reason we use height is that the number of elliptic curves with height 
$\leqslant X$ is $C X^{5/6}+(\text{error term})$ for some constant $C$. On the 
other hand, we have no idea how many elliptic curves there are with bounded 
conductor. So even though height isn't as conceptually elegant, we'll order 
elliptic curves by it. 

Given $E_{/\dQ}$ with algebraic rank $3$, we could numerically show that the 
analytic rank is $\leqslant 3$, then use sign considerations to get t
$r_\mathrm{an}\in \{1,3\}$. If $r_\mathrm{an}=1$, we know that 
$r_\mathrm{alg}=1$, a contradiction, hence $r_\mathrm{an}=3$. Thus, for 
(specific) elliptic curves with algebraic rank $3$, it is possible to prove 
BSD. At the present, no one knows how to do this for a family of elliptic 
curves. From the Kummer exact sequence $0\to E[p] \to E\xrightarrow p E \to 0$, 
one gets a short exact sequence
\[\begin{tikzcd}
  0 \ar[r] 
    & E(\dQ)/p \ar[r] 
    & \h^1(\dQ,E[p]) \ar[r] 
    & \h^1(\dQ,E)[p] 
    & 0 .
\end{tikzcd}\]
One has a similar short exact sequence for each place $v$. This yields a 
(now-familiar) commutative diagram:
\[\begin{tikzcd}
  0 \ar[r] 
    & E(\dQ)/p \ar[r] \ar[d] 
    & \h^1(\dQ,E[p]) \ar[r] \ar[d] 
    & \h^1(\dQ,E)[p] \ar[d] \ar[r] 
    & 0 \\
  0 \ar[r] 
    & \prod E(\dQ_v)/p \ar[r] 
    & \prod \h^1(\dQ_v,E[p]) \ar[r] 
    & \prod \h^1(\dQ_v,E)[p] \ar[r] 
    & 0.
\end{tikzcd}\]
Even better, we can write 
\[
  \selmer_p(E)  \ker\left(\h^1(\dQ,E[p])\to \h^1(\dA,E)[p]\right) ,
\]
where $\dA$ is the ring of adeles over $\dQ$ and ``$\h^1$'' denotes \'etale 
cohomology. 

We'd like to give a geometric interpretation of $\h^1(\dQ,E)$ and 
$\h^1(\dQ,E[p])$. First, note that 
$\dim (E(\dQ)/p)=\rank(E)+\delta$, where $\delta=\dim E[p](\dQ)\leqslant 2$. 
With respect to height, $\delta>0$ on a density zero set, so we may as well 
assume $\delta=0$. Recall there is a short exact sequence 
\[\begin{tikzcd}
  0 \ar[r] 
    & E(\dQ)/p \ar[r] 
    & \selmer_p(E) \ar[r] 
    & \sha[p] \ar[r] 
    & 0 .
\end{tikzcd}\]
Separating $\selmer_p(E)$ into its contributions from $E(\dQ)/p$ and 
$\sha[p]$ is very difficult. One conjectures that 
$\sha=\ker(\h^1(\dQ,E) \to \h^1(\dA,E))$ is finite for all $E$. This conjecture 
implies that $\sha[p]=0$ for almost all $p$. Cassels has constructed a perfect 
pairing $\sha\times \sha\to \dQ/\dZ$. So if $\#\sha<\infty$, then $\#\sha$ 
should be a perfect square. The ``$\sha$-part'' of BSD can be computed in 
some examples, and it always turns out to be a perfect square. 

[\ldots recap of strategy for Bhargava's proof\ldots]
\[
  \rho:\Gamma_\dQ\to \operatorname{GL}_2(\mathbf T_\mathfrak m) .
\]





\subsection{Group cohomology}

For the moment, let $G$ be a profinite group. By a $G$-module, we mean 
an abelian group $M$ with $G$-action, such that for all $m\in M$, the 
stabilizer $\stabilizer_G(m)$ is an open subgroup of $G$. The category of 
$G$-modules is an abelian category with enough injectives. Indeed, let 
$\cC$ be the category of finite sets with continuous $G$-action. Then the 
category of $G$-modules is equivalent to the category of sheaves of abelian 
groups on $\cC$. It is known in general that the category of abelian group 
objects in a Grothendieck topos has enough injectives. 

\begin{definition}
Let $M$ be a $G$-module. We define $\h^i(G,M)=\mathsf R^i (-)^G(M)$. 
\end{definition}

That is, $\h^i(G,-)$ is the $i$-th derived functor of $(-)^G$. The group 
$\h^2(G,M)$ classifies central extensions $0\to M\to X\to G\to 1$, up to 
isomorphism making the following diagram commute:
\[\begin{tikzcd}
  1 \ar[r] 
    & M \ar[r] \ar[d, equal] 
    & X \ar[r] \ar[d, "\wr"] 
    & G \ar[r] \ar[d, equal] 
    & 1 \\
  1 \ar[r] 
    & M \ar[r] 
    & Y \ar[r] 
    & G \ar[r] 
    & 1 .
\end{tikzcd}\]
For $i\leqslant 2$, $\h^i(G,M)$ can be more readily computed directly. Let 
\begin{align*}
  Z^1(G,M) &= \{f:G\to M\text{ continuous, such that }f(\sigma\tau)=f(\sigma)+\sigma(f(\tau))\} \\
  B^1(G,M) &= \{f:G\to M\text{ of the form }\sigma\mapsto \sigma(m)-m\text{ for some }m\in M\} .
\end{align*}
Then $\h^1(G,M)\simeq Z^1(G,M)/B^1(G,M)$. 

If $H\subset G$, the restriction functor $\res_H^G(-)$ from $G$-modules to 
$H$-modules is exact, so the natural inclusion 
$\h^0(H,\res_H^G M)\monic \h^0(G,M)$ extends canonically to a natural 
transformation $\h^\bullet(H,\res_H^G(-))\to \h^\bullet(G,-)$. 

\begin{theorem}
If $G$ acts trivially on $M$, then $\h^1(G,M)=\hom(G,M)$. 
\end{theorem}
\begin{proof}
If $G$ acts trivially, then $B^1(G,M)=0$. Also, $Z^1(G,M)$ consists of those 
$f:G\to M$ such that $f(\sigma\tau)=f(\sigma)+f(\tau)$. 
\end{proof}

Let $k$ be a number field. If $v$ is a place of $k$, write $G_v\subset G_k$ 
for the Galois group $\galois(\overline{k_v}/k_v)$. If $M$ is a $G_k$-module 
$v$ is a place of $k$, one puts 
\[
  \sha(M) = \ker\left(\h^1(k,M) \to \prod_v \h^1(k_v,M)\right) .
\]

\begin{theorem}
Let $k$ be a number field, $M$ a finite $G_k$-module. Then 
$\sha(M)$ is finite. 
\end{theorem}
\begin{proof}
Let $K=k(M)$; then $G_K$ acts trivially on $M$. If $f\in \sha(M)$, then 
$f|_{G_K}$ is a homomorphism cutting out an extension $L/K$. Since 
$f|_{G_v}=0$ for all places $v$ of $K$, we know that all primes in $L$ split 
completely (hence are unramified) over $K$. By the Hermite-Minkowski thorem, 
there are only finitely many possibilities for $f$. 
\end{proof}

Via the above proof, we see that cohomology relates to class groups (via 
Hermite-Minkowski). Let $M$ be a finite $\dF_p[G_{\dQ_l}]$-module. Put 
$M^\ast=\hom(M,\dmu_p)$. Then there is a 
perfect pairing 
$\h^\bullet(\dQ_l,M)\times \h^{2-\bullet}(\dQ_l,M^\ast)\to \h^2(\dQ_l,\dmu_p)$, 
and 
\[
  \sum_{i=0}^2 (-1)^i\dim_{\dF_p} \h^i(\dQ_l,M) = \begin{cases} 0 & l\ne p \\ -\dim(M) & l = p \end{cases}
\]
This is \emph{Local Tate Duality}. It should be thought of as analogous 
to algebraic topology. In fact, \'etale cohomology is the ``right'' cohomology 
theory for varieties, and satisfies a good duality theory that resembles 
Poincar\'e duality. There is a duality theory for modules over $G_k$ when $k$ 
is a global field. You can compute 
$\dim \h^1(\dQ,M)-\dim \h^2(\dQ,M)$ when $M$ is a $\dF_p[G_\dQ]$-module. 
Getting either one of $\h^1$ or $\h^2$ involves computing the class group of 
$\dQ(M)$. 





\subsection{Geometric interpretation of cohomology}

The elliptic curves 
\begin{align*}
  E_1 &: y^2=x^3+x+1 \\
  E_2 &: 3 y^2=x^3+x+1 ,
\end{align*}
are isomorphic over $\dQ(\sqrt 3)$ but not over $\dQ$. Considering such pairs 
is a standard approach to the $r_\mathrm{an}\leqslant 1$ cases of BSD. 

In general, if $L/K$ is a field extension and $X_{/L}$ is some ``arithmetic 
object'' the set of $Y_{/K}$ giving rise to $X$ via base-change is classified 
by $\h^1(\galois(L/K),\automorphisms X)$. Since $\automorphisms(X)$ might be 
nonabelian, this $\h^1(L/K,\automorphisms X)$ may only be a pointed set. 

Let $E_{/\dQ}$ be an elliptic curve. We'll realise $E[p]$ and 
$E(\overline\dQ)$ as automorphisms of pairs of objects, one entry of which is 
$E$. So $\h^1(\dQ,E[p])$ and $\h^1(\dQ,E)$ will classify descents of some type 
of object to $\dQ$. 

\begin{definition}
Let $G$ be a profinite group, $N$ a (possibly nonabelian) group with 
continuous $G$-action. Then $Z^1(G,N)$ is the set of continuous 
$f:G\to N$ such that $f(\sigma\tau)=f(\sigma)\sigma(f(\tau))$ for all 
$\sigma,\tau\in G$. If $f,g\in Z^1(G,N)$, we say they are 
\emph{cohomologous} if there exists $n\in N$ such that 
$f(\sigma)=n^{-1} g(\sigma) \sigma(n)$ for all $\sigma\in G$. Let 
$\h^1(G,N)=Z^1(G,N)/\sim$, where $f\sim g$ if $f$ is cohomologous to $g$. 
\end{definition}

We'll find elements of $\h^1(\dQ,E[p])$ that correspond bijectively to 
pairs $(C,D)$ where $C$ is a curve of genus $1$ over $\dQ$ and $D$ is a 
divisor of degree $p$ over $\dQ$. These correspond to maps 
$C\to S\simeq_{\overline\dQ} \dP^{p-1}$, where $C\simeq_{\overline\dQ} E$. 

\begin{example}
Let $C_{/\dQ}$ be the projective curve $3 x^3+4 y^3+5 z^3=0$. Then 
$C(\dQ)=\varnothing$. However, if we take some 
$x\in C(\overline\dQ)$ and let $D=\sum_{\sigma\in G_\dQ} (\sigma(x))$, then 
$D$ is a divisor on $C$ defined over $\dQ$. Essentially, the point is that 
even if $C(\dQ)=\varnothing$, we will have 
$\picard^\circ(C)(\dQ)\ne\varnothing$. 
\end{example}

\begin{theorem}
Let $E_{/\dQ}$ be an elliptic curve. Then $\selmer_p(E)$ is finite. 
\end{theorem}
\begin{proof}
We essentially reprove weak Mordell-Weil. Recall that if $l\ne p$ is a prime of 
good reduction, then $E[p]\monic E_{\dF_l}(\overline{\dF_l})$. Alternatively, 
the inertia group $I_l$ acts trivially on $E[p]$. So we'll prove that any 
$f\in \selmer_p E$ is unramified outside a (fixed) finite set, namely $p$ 
together with the set of bad primes for $E$. Let $S$ be that set. For 
$l\notin S$, the restriction $f|_{G_{\dQ_l}}$ is a coboundary in 
$\h^1(\dQ_l,E)$. So there exists $P\in E(\overline{\dQ_l})$ such that 
$f(\sigma)=\sigma(P)-P$ for all $\sigma\in G_{\dQ_l}$. But 
$\sigma(P)-P\in E[p]$. For $\sigma\in I_l$, $\sigma(P)-P=0$ in 
$E(\overline{\dF_l})$. Since $E[p]\monic E_{\dF_l}[p]$, we have 
$\sigma(P)=P$, i.e.~$f(I_l)=0$. We have shown that all classes 
$f\in \selmer_p(E)$ are unramified outside of $S$, which yields the result. 
\end{proof}

This proof depends on the finiteness of the class group. 

\begin{lemma}
Let $E_{/\dQ}$ be an elliptic curve, $\phi:E\to E$ a nonconstant map such 
that $\phi(0)=0$. Then $\phi$ is an isogeny. Moreover, if $\phi$ is an 
isomorphism ad commutes with all translations, then $\phi=1$. 
\end{lemma}
\begin{proof}
We sketch a proof of the latter claim. If $E$ is non-CM, then 
$\phi=\pm 1$. Clearly $-1$ does not commute with translations, so $\phi=1$. If 
$E$ has CM, think about $E_{/\dC}\simeq \dZ/\Lambda$ for $\Lambda$ an order 
in a quadratic imaginary field. Say $\Lambda=\langle 1,\tau\rangle$ for 
$\tau\in \dQ(\sqrt{-d})$ where $d\in \{1,3\}$. The only automorphisms are 
$\{\pm 1,\pm i\}$ or $\{\pm 1,\pm \omega,\pm\omega^2\}$. You can check 
directly that none of these commute with translations, except for $1$. 

Alternatively, just plug $x=0$ into $\phi(x+y)=\phi(x)+y$. 
\end{proof}

\begin{definition}
Let $k$ be a field of characteristic zero, $C_{/k}$ a curve. A \emph{twist} of 
$C$ is a curve $D_{/k}$ together with an isomorphism 
$\phi:C_{\overline k}\iso D_{\overline k}$. 
\end{definition}

The philosophy is that $\h^1(G_\dQ,\automorphisms(E_{\overline\dQ}))$ 
classifies twists of $E$ over $\dQ$. More generally, if ``$X_{/\dQ}$ is an 
object,'' then $\h^1(G_\dQ,\automorphisms(X_{\overline\dQ}))$ classifies 
twists of $X$.''

\begin{theorem}
Let $X_{/\dQ}$ be a twist of $\dP_{/\dQ}^n$. Then $X\simeq \dP^n$ if and only 
if $X(\dQ)\ne\varnothing$. Moreover, if $X(\dA)\ne\varnothing$, then 
$X(\dQ)\ne\varnothing$. 
\end{theorem}

We will use this theorem because $\alpha\in \selmer_p E$ gives rise to a map 
$C\to S$, where $C$ is a twist of $E$ and $S$ is a twist of 
$\dP^{p-1}_{/\dQ}$. But elements of $\selmer_p(E)$ are locally trivial. This 
will imply $C(\dA)\ne\varnothing$. Thus $S(\dA)\ne\varnothing$, so 
$S\simeq \dP^{p-1}$. We end up with a curve $C\subset \dP^{p-1}$. We hope to 
write such curves as the zero-locus of a single function, or a small 
managable set of functions. For $p\in \{2,3,5\}$, this can be done. 

Suppose we are given a twist $C'$ of $C_{/\dQ}$ with 
$\phi:C'_{\overline\dQ}\iso C_{\overline\dQ}$. Define 
$f:G_\dQ\to \automorphisms(C_{\overline\dQ})$ by 
$f(\sigma)=\phi^\sigma\circ\phi^{-1}$. Here, $\phi^\sigma(x)=\sigma(\phi(x))$. 

\begin{proposition}
As constructed above, $f$ is a $1$-cocycle. Moreover, adjusting $f$ by a 
$1$-coboundary preserves the isomorphism class of $C'$. So $C\mapsto f$ induces 
a bijection between the set of isomorphism classes of twists of $C$ and 
$\h^1(G_\dQ,\automorphisms(C_{\overline\dQ}))$. 
\end{proposition}
\begin{proof}
We begin with a computation: 
\begin{align*}
  f(\sigma\tau) &= \phi^{\sigma\tau}\phi^{-1} \\ 
    &= \phi^{\sigma\tau}(\phi^\tau)^{-1}\phi^\tau\phi^{-1} \\
    &= (\phi^\sigma\phi^{-1})^\tau \phi^\tau \phi^{-1} \\
    &= \prescript{\tau}{}{f(\sigma)} f(\tau) .
\end{align*}
The rest is relatively trivial. Showing surjectivity is a bit tricky -- this 
ends up being a special case of descent (\'etale descent, to be precise). 
\end{proof}

We'll begin with the easier case of a discrete $G_\dQ$-module. Let 
$f\in \h^1(G_\dQ,\automorphisms(M))$; we need to build a twist of $M$. 
Put $\sigma\cdot_f m = f(\sigma)(\sigma(m))$. Check that $\cdot_f$ gives 
a well-defined action of $G_\dQ$ on $M$. Write $f^\ast M$ for the twist of $M$ 
that we have defined. Check that if $f$ is a coboundary, then 
$M\simeq f^\ast M$. 

\begin{theorem}[Bhargava-Shankar]
Set $E:y^2=x^3-\frac{I}{3} x-\frac{J}{27}$ for $I,J\in k$. Then 
$E(k)/2$ is naturally in bijection with $\PGL_2$-orbits of $k$-soluble 
binary quartic forms having invariants $I$ and $J$, via 
\[
  (\varepsilon,\eta)\mapsto \frac{x^4}{4}-\frac 3 2 \varepsilon x^2 y^2 + 2\eta x y^3 + \left(\frac I 3-\frac 3 4 \varepsilon^2\right) y^4 .
\]
\end{theorem}





\subsection{Torsors}

\begin{definition}
Let $E_{/\dQ}$ be an elliiptic curve. A \emph{torsor} under $E$ is a curve 
$C_{/\dQ}$ with $E$-action such that $C_{\overline\dQ}\simeq E_{\overline\dQ}$ 
as $E$-varieties. 
\end{definition}

That is, we are given a map $\mu:E\times C\to C$ defined over $\dQ$ satisfying 
the appropriate associativity conditions. Given any $x\in C(\overline\dQ)$, the 
map $\mu(-,x):E_{\overline\dQ}\to C_{\overline\dQ}$ will be an isomorphism of 
varieties over $\overline\dQ$. 

\begin{definition}
Two $E$-torsors $C_1,C_2$ are isomorphic if there exists an $E$-equivariant 
isomorphism $\phi:C_1\iso C_2$ defined over $\dQ$. 
\end{definition}

So we need the following diagram to commute:
\[\begin{tikzcd}
  E\times C_1 \ar[r, "\mu_1"] \ar[d, "1\times \phi"] 
    & C_1 \ar[d, "\phi"] \\
  E\times C_2 \ar[r, "\mu_2"] 
    & C_2 .
\end{tikzcd}\]

The \emph{trivial torsor} is $E\times E\xrightarrow+ E$. 
Note that by definition, every torsor is a twist of the trivial torsor. Recall 
that $\automorphisms_E(E)$, the group of $E$-equivariant automorphisms of $E$, 
is $E$. Thus $\h^1(\dQ,E)$ classifies $E$-torsors up to isomorphism. 
In general, let $G_{/k}$ be a group scheme. Then $\h^1(\dQ,G)$ classifies 
$G$-torsors up to isomorphism. 

The papers \cite{cfnss-2008,cfnss-2009,cfnss-2014} give many different 
descriptions of elements of $\h^1(\dQ,E[p])$ for $p$ not necessarily prime. 

\begin{definition}
A torsor divisor class pair (of degree $p$) is an $E$-torsor with a divisor 
$D$ of degree $p$, defined over $\dQ$. 
\end{definition}

Two torsor divisor class pairs $(C_1,D_1)$, $(C_2,D_2)$ are isomorphic if there 
is an isomorphism $\phi:C_1\to C_2$ of $E$-torsors such that 
$\phi^\ast D_2\sim D_1$ in $\picard(C_1)$. The trivial torsor divisor class 
pair is $(E,p\cdot 0)$. 

\begin{lemma}
Every torsor divisor class pair is a twist of $(E,p\cdot 0)$. 
\end{lemma}
\begin{proof}
Let $\phi:C_{\overline\dQ}\iso E_{\overline\dQ}$. We need 
$\phi^\ast(p\cdot 0)\sim 0$. Equivilantly, we need 
$p\cdot 0-(\phi^{-1})^\ast D\sim 0$. It is sufficient to show that the sum of 
points (with multiplicities) in $\phi^{-1\ast} D$ equal to $0$ on $E$? Say 
the sum is $x\in E(\overline\dQ)$. Let $p y=x$ and compose $\phi$ with 
``translate by $-y$.'' This gives the result. 
\end{proof}

\begin{lemma}
$\automorphisms_{E_{\overline\dQ}}(E,p\cdot 0) = E[p]$. 
\end{lemma}
\begin{proof}
Any automorphism of a torsor divisr class pair must be translation by some 
element, and the condition $\phi\ast D_2\sim D_1$ forces that point to be 
$p$-torsion. 
\end{proof}

It follows that $\h^1(\dQ,E[p])$ is in bijection with the set of isomorphism 
classes of torsor divisor class pairs. Note that 
$\h^1(\dQ,E[p])\supset \selmer_p E\supset E(\dQ)/p$. The latter set corresponds 
to those $(C,D)$ for which $C(\dQ)\ne\varnothing$, while $\selmer_p E$ 
corresponds to the set of $(C,D)$ for which $C(\dA)\ne\varnothing$. 

\begin{definition}
A \emph{Brauer-Severi diagram} $C\to S$ is a morphism from $C$, an $E$-torsor, 
to $S$, a twist of $\dP^{p-1}$. 
\end{definition}

Twoo Brauer-Severi diagrams $C_1\to S_1$ and $C_2\to S_2$ are isomorphic if 
there isomorphisms $\phi:C_1\iso C_2$ and $\psi:S_1\to S_2$ making the 
following diagram commute:
\[\begin{tikzcd}
  C_1 \ar[r] \ar[d, "\phi"] 
    & S_1 \ar[d, "\psi"] \\
  C_2 \ar[r] 
    & S_2 .
\end{tikzcd}\]
The trivial Brauer-Severi diagram is the map $E\to \dP^{p-1}$ coming from 
the divisor $p\cdot 0$. 

\begin{lemma}
The trivial Brauer-Severi diagram $E\to \dP^{p-1}$ satisfies 
$\automorphisms(E\to \dP^{p-1})=E[p]$. 
\end{lemma}
\begin{proof}
Let $\phi:E_{\overline\dQ}\iso E_{\overline\dQ}$. We need 
$\phi^\ast(p\cdot 0)=(p\cdot 0)$. We know $\phi=(-)+x$ for some $x$, and 
$\phi^\ast(p\cdot 0)\sim (p\cdot 0)$ implies $x\in E[p]$. We need an 
automorphism of projective space making the following diagram commute:
\[\begin{tikzcd}
  E \ar[r, "p\cdot 0"] \ar[d, "+x"] 
    & \dP^{p-1} \ar[d, dashrightarrow] \\
  E \ar[r, "\phi^\ast(p\cdot 0)"] 
    & \dP^{p-1} .
\end{tikzcd}\]
Such an automorphism exists because $\automorphisms(\dP^{p-1})=\GL(p)$ acts 
transitively on $\dP^{p-1}$. 
\end{proof}

\begin{definition}
An \emph{$n$-covering} of $E$ is a curve $C_{/\dQ}$ together with 
$\pi:C\to E$, such that there is an isomorphism 
$\phi:C_{\overline\dQ}\iso E_{\overline\dQ}$ making the following diagram 
commute: 
\[\begin{tikzcd}
  C \ar[d, "\phi"] \ar[dr, "\pi"] \\
  E \ar[r, "n"] 
    & E .
\end{tikzcd}\]
\end{definition}

The trivial $n$-covering is $\cdot n:E\to E$. 

\begin{lemma}
All $n$-coverings are twists of the trivial $n$-covering. Moreover, 
the trivial $n$-covering has automorphism group $E[n]$. 
\end{lemma}

Thus isomorphism classes of $n$-coverings of $E$ are in bijection with 
$\h^1(\dQ,E[n])$. 

\begin{definition}
Let $\pi:C\epic E$ be an $n$-covering. Then $(C,\pi)$ is \emph{solvable} 
if $C(\dQ)\ne\varnothing$, and \emph{locally soluble} if 
$C(\dA)\ne\varnothing$. 
\end{definition}

Let $x_0\in E(\dQ)/p\subset \selmer_p E\subset \h^1(\dQ,E[p])$. Consider 
$\pi(x)=p\cdot x+x_0$. Then $\phi(x)=x+(\text{some }\frac 1 p x_0)$. So 
$\pi$ is trivial if and only if $x_0=0$ in $E(\dQ)/p$. 

Let $\alpha\in \selmer_p(E)$. We know that $\alpha=0$ in 
$\h^1(\dA,E)$. Thus, if $C$ is the corresponding $p$-covering, we have 
$C(\dA)\ne\varnothing$. Finally, $\alpha$ will correspond to a Brauer-Severi 
embedding which is locally soluble. Since projective spaces satisfy the 
Hasse principle, locally soluble Brauer-Severi embeddings look like 
$C\to \dP^{p-1}$. 




