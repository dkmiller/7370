% !TEX root = 7370.tex





\section{Main stuff}

Let $k$ be a number field, $C_{/k}$ a smooth proper curve. Let $g$ be the 
genus of $C$. If $g=0$, either $C(k)=\varnothing$, or $C\simeq \dP^1_{/k}$. 
If $g>1$, then Faltings gave an ineffective proof that $\# C(k)<\infty$. 
If $g=1$, then either $C(k)=\varnothing$, or, if $C(k)\ne\varnothing$, then 
$C(k)$ is a finitely generated abelian group. Recall that the Birch and 
Swinnerton-Dyer conjecture is that $\rank(E(k))=\order_{s=1} L(E,s)$. The 
function $L(E,s)$ satisfies a functional equation $L(E,s) \sim \pm L(E,2-s)$. 
We expect $(-1)^{\rank(E)}=\pm$ accordingly. This is the ``sign conjecture.'' 
Kolyvagin and others have proved that if $L(E,s)$ has a pole of order at most 
$1$ at $1$, then BSD holds for $E$. A folklore question is: is 
$\sup_{E/\dQ} \rank(E)<\infty$?

Any $E_{/\dQ}$ can be written as $y^2=x^3+A x+B$ where $A,B\in \dZ$, and 
$p^6\mid B\Rightarrow p^4\nmid A$. The \emph{height} of $E$ is 
$H(E)=\max(4|A|^3,27 B^2)$. This is asymptotically degree-six in the roots 
of $x^3+A x+B$. We could have ordered $E$ by (Galois-theoretic) conductor. The 
reason we use height is that the number of elliptic curves with height 
$\leqslant X$ is $C X^{5/6}+(\text{error term})$ for some constant $C$. On the 
other hand, we have no idea how many elliptic curves there are with bounded 
conductor. So even though height isn't as conceptually elegant, we'll order 
elliptic curves by it. 

Given $E_{/\dQ}$ with algebraic rank $3$, we could numerically show that the 
analytic rank is $\leqslant 3$, then use sign considerations to get t
$r_\mathrm{an}\in \{1,3\}$. If $r_\mathrm{an}=1$, we know that 
$r_\mathrm{alg}=1$, a contradiction, hence $r_\mathrm{an}=3$. Thus, for 
(specific) elliptic curves with algebraic rank $3$, it is possible to prove 
BSD. At the present, no one knows how to do this for a family of elliptic 
curves. From the Kummer exact sequence $0\to E[p] \to E\xrightarrow p E \to 0$, 
one gets a short exact sequence
\[\begin{tikzcd}
  0 \ar[r] 
    & E(\dQ)/p \ar[r] 
    & \h^1(\dQ,E[p]) \ar[r] 
    & \h^1(\dQ,E)[p] 
    & 0 .
\end{tikzcd}\]
One has a similar short exact sequence for each place $v$. This yields a 
(now-familiar) commutative diagram:
\[\begin{tikzcd}
  0 \ar[r] 
    & E(\dQ)/p \ar[r] \ar[d] 
    & \h^1(\dQ,E[p]) \ar[r] \ar[d] 
    & \h^1(\dQ,E)[p] \ar[d] \ar[r] 
    & 0 \\
  0 \ar[r] 
    & \prod E(\dQ_v)/p \ar[r] 
    & \prod \h^1(\dQ_v,E[p]) \ar[r] 
    & \prod \h^1(\dQ_v,E)[p] \ar[r] 
    & 0.
\end{tikzcd}\]
Even better, we can write 
\[
  \selmer_p(E)  \ker\left(\h^1(\dQ,E[p])\to \h^1(\dA,E)[p]\right) ,
\]
where $\dA$ is the ring of adeles over $\dQ$ and ``$\h^1$'' denotes \'etale 
cohomology. 

We'd like to give a geometric interpretation of $\h^1(\dQ,E)$ and 
$\h^1(\dQ,E[p])$. First, note that 
$\dim (E(\dQ)/p)=\rank(E)+\delta$, where $\delta=\dim E[p](\dQ)\leqslant 2$. 
With respect to height, $\delta>0$ on a density zero set, so we may as well 
assume $\delta=0$. Recall there is a short exact sequence 
\[\begin{tikzcd}
  0 \ar[r] 
    & E(\dQ)/p \ar[r] 
    & \selmer_p(E) \ar[r] 
    & \sha[p] \ar[r] 
    & 0 .
\end{tikzcd}\]
Separating $\selmer_p(E)$ into its contributions from $E(\dQ)/p$ and 
$\sha[p]$ is very difficult. One conjectures that 
$\sha=\ker(\h^1(\dQ,E) \to \h^1(\dA,E))$ is finite for all $E$. This conjecture 
implies that $\sha[p]=0$ for almost all $p$. Cassels has constructed a perfect 
pairing $\sha\times \sha\to \dQ/\dZ$. So if $\#\sha<\infty$, then $\#\sha$ 
should be a perfect square. The ``$\sha$-part'' of BSD can be computed in 
some examples, and it always turns out to be a perfect square. 

[\ldots recap of strategy for Bhargava's proof\ldots]
\[
  \rho:\Gamma_\dQ\to \operatorname{GL}_2(\mathbf T_\mathfrak m) .
\]





\subsection{Group cohomology}

For the moment, let $G$ be a profinite group. By a $G$-module, we mean 
an abelian group $M$ with $G$-action, such that for all $m\in M$, the 
stabilizer $\stabilizer_G(m)$ is an open subgroup of $G$. The category of 
$G$-modules is an abelian category with enough injectives. Indeed, let 
$\cC$ be the category of finite sets with continuous $G$-action. Then the 
category of $G$-modules is equivalent to the category of sheaves of abelian 
groups on $\cC$. It is known in general that the category of abelian group 
objects in a Grothendieck topos has enough injectives. 

\begin{definition}
Let $M$ be a $G$-module. We define $\h^i(G,M)=\mathsf R^i (-)^G(M)$. 
\end{definition}

That is, $\h^i(G,-)$ is the $i$-th derived functor of $(-)^G$. The group 
$\h^2(G,M)$ classifies central extensions $0\to M\to X\to G\to 1$, up to 
isomorphism making the following diagram commute:
\[\begin{tikzcd}
  1 \ar[r] 
    & M \ar[r] \ar[d, equal] 
    & X \ar[r] \ar[d, "\wr"] 
    & G \ar[r] \ar[d, equal] 
    & 1 \\
  1 \ar[r] 
    & M \ar[r] 
    & Y \ar[r] 
    & G \ar[r] 
    & 1 .
\end{tikzcd}\]
For $i\leqslant 2$, $\h^i(G,M)$ can be more readily computed directly. Let 
\begin{align*}
  Z^1(G,M) &= \{f:G\to M\text{ continuous, such that }f(\sigma\tau)=f(\sigma)+\sigma(f(\tau))\} \\
  B^1(G,M) &= \{f:G\to M\text{ of the form }\sigma\mapsto \sigma(m)-m\text{ for some }m\in M\} .
\end{align*}
Then $\h^1(G,M)\simeq Z^1(G,M)/B^1(G,M)$. 

If $H\subset G$, the restriction functor $\res_H^G(-)$ from $G$-modules to 
$H$-modules is exact, so the natural inclusion 
$\h^0(H,\res_H^G M)\monic \h^0(G,M)$ extends canonically to a natural 
transformation $\h^\bullet(H,\res_H^G(-))\to \h^\bullet(G,-)$. 

\begin{theorem}
If $G$ acts trivially on $M$, then $\h^1(G,M)=\hom(G,M)$. 
\end{theorem}
\begin{proof}
If $G$ acts trivially, then $B^1(G,M)=0$. Also, $Z^1(G,M)$ consists of those 
$f:G\to M$ such that $f(\sigma\tau)=f(\sigma)+f(\tau)$. 
\end{proof}

Let $k$ be a number field. If $v$ is a place of $k$, write $G_v\subset G_k$ 
for the Galois group $\galois(\overline{k_v}/k_v)$. If $M$ is a $G_k$-module 
$v$ is a place of $k$, one puts 
\[
  \sha(M) = \ker\left(\h^1(k,M) \to \prod_v \h^1(k_v,M)\right) .
\]

\begin{theorem}
Let $k$ be a number field, $M$ a finite $G_k$-module. Then 
$\sha(M)$ is finite. 
\end{theorem}
\begin{proof}
Let $K=k(M)$; then $G_K$ acts trivially on $M$. If $f\in \sha(M)$, then 
$f|_{G_K}$ is a homomorphism cutting out an extension $L/K$. Since 
$f|_{G_v}=0$ for all places $v$ of $K$, we know that all primes in $L$ split 
completely (hence are unramified) over $K$. By the Hermite-Minkowski thorem, 
there are only finitely many possibilities for $f$. 
\end{proof}

Via the above proof, we see that cohomology relates to class groups (via 
Hermite-Minkowski). Let $M$ be a finite $\dF_p[G_{\dQ_l}]$-module. Put 
$M^\ast=\hom(M,\dmu_p)$. Then there is a 
perfect pairing 
$\h^\bullet(\dQ_l,M)\times \h^{2-\bullet}(\dQ_l,M^\ast)\to \h^2(\dQ_l,\dmu_p)$, 
and 
\[
  \sum_{i=0}^2 (-1)^i\dim_{\dF_p} \h^i(\dQ_l,M) = \begin{cases} 0 & l\ne p \\ -\dim(M) & l = p \end{cases}
\]
This is \emph{Local Tate Duality}. It should be thought of as analogous 
to algebraic topology. In fact, \'etale cohomology is the ``right'' cohomology 
theory for varieties, and satisfies a good duality theory that resembles 
Poincar\'e duality. There is a duality theory for modules over $G_k$ when $k$ 
is a global field. You can compute 
$\dim \h^1(\dQ,M)-\dim \h^2(\dQ,M)$ when $M$ is a $\dF_p[G_\dQ]$-module. 
Getting either one of $\h^1$ or $\h^2$ involves computing the class group of 
$\dQ(M)$. 





\subsection{Geometric interpretation of cohomology}

The elliptic curves 
\begin{align*}
  E_1 &: y^2=x^3+x+1 \\
  E_2 &: 3 y^2=x^3+x+1 ,
\end{align*}
are isomorphic over $\dQ(\sqrt 3)$ but not over $\dQ$. Considering such pairs 
is a standard approach to the $r_\mathrm{an}\leqslant 1$ cases of BSD. 

In general, if $L/K$ is a field extension and $X_{/L}$ is some ``arithmetic 
object'' the set of $Y_{/K}$ giving rise to $X$ via base-change is classified 
by $\h^1(\galois(L/K),\automorphisms X)$. Since $\automorphisms(X)$ might be 
nonabelian, this $\h^1(L/K,\automorphisms X)$ may only be a pointed set. 

Let $E_{/\dQ}$ be an elliptic curve. We'll realise $E[p]$ and 
$E(\overline\dQ)$ as automorphisms of pairs of objects, one entry of which is 
$E$. So $\h^1(\dQ,E[p])$ and $\h^1(\dQ,E)$ will classify descents of some type 
of object to $\dQ$. 

\begin{definition}
Let $G$ be a profinite group, $N$ a (possibly nonabelian) group with 
continuous $G$-action. Then $Z^1(G,N)$ is the set of continuous 
$f:G\to N$ such that $f(\sigma\tau)=f(\sigma)\sigma(f(\tau))$ for all 
$\sigma,\tau\in G$. If $f,g\in Z^1(G,N)$, we say they are 
\emph{cohomologous} if there exists $n\in N$ such that 
$f(\sigma)=n^{-1} g(\sigma) \sigma(n)$ for all $\sigma\in G$. Let 
$\h^1(G,N)=Z^1(G,N)/\sim$, where $f\sim g$ if $f$ is cohomologous to $g$. 
\end{definition}

We'll find elements of $\h^1(\dQ,E[p])$ that correspond bijectively to 
pairs $(C,D)$ where $C$ is a curve of genus $1$ over $\dQ$ and $D$ is a 
divisor of degree $p$ over $\dQ$. These correspond to maps 
$C\to S\simeq_{\overline\dQ} \dP^{p-1}$, where $C\simeq_{\overline\dQ} E$. 

\begin{example}
Let $C_{/\dQ}$ be the projective curve $3 x^3+4 y^3+5 z^3=0$. Then 
$C(\dQ)=\varnothing$. However, if we take some 
$x\in C(\overline\dQ)$ and let $D=\sum_{\sigma\in G_\dQ} (\sigma(x))$, then 
$D$ is a divisor on $C$ defined over $\dQ$. Essentially, the point is that 
even if $C(\dQ)=\varnothing$, we will have 
$\picard^\circ(C)(\dQ)\ne\varnothing$. 
\end{example}




