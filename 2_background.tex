% !TEX root = 7370.tex

\section{Background}

We'll thread a very narrow path to the main theorems, assuming a bunch of 
facts along the way. A good source on the algebraic geometry we'll do is
\cite[ch.2,3]{silverman-2009}. 





\subsection{Riemann-Roch}

We'll spend the rest of this subsection explaining the terms in the following 
theorem. 

\begin{theorem}[Riemann-Roch]
Let $C_{/k}$ be a smooth projective curve. Let $K$ be the canonical divisor and 
$D$ a divisor. Then $l(D)-l(K-D)=\deg D-g+1$. 
\end{theorem}

For us, a curve $C_{/k}$ is the zero locus of a (non-constant) polynomial 
$f\in k[x,y]$. We stress the fact that $k$ is \emph{not} necessarily 
algebraically closed here. Pictorially, smoothness means that $C$ has no 
cusps or self-intersections. For example, $f$ can't be something like 
$x y$ or $x^2-y^3$. More formally, we want the tangent space to be 
$1$-dimensional at every point on $C$. For any $c\in C$, we have a 
matrix 
$\begin{pmatrix} \frac{\partial f}{\partial x}(c),\frac{\partial f}{\partial y}(c)\end{pmatrix}$
which needs to be nondegenerate, i.e.~the partials of $f$ never 
simultaneously vanish. Note that here, ``$c\in C$'' means $c\in C(\bar k)$, so 
we consider points not defined over $k$. We'll constantly move between the 
affine curve $V(f)\subset \dA^2$ and the projective curve in $\dP^2$ cut out by 
the projectivisation of $f$. 

A \emph{divisor} on a curve $C$ is a formal $\dZ$-linear combination of points 
(defined over $\bar k$). 

\begin{example}
Let $C:x^2+y^2=10$. Let $P=(4,i\sqrt 6)$ and $Q=(4,-i\sqrt 6)$. Then $P$ and 
$Q$ are not defined over $\dQ$, but $P+Q$ is a divisor on $C$ defined over 
$\dQ$. 
\end{example}

The absolute Galois group $G_k$ of $k$ acts on the space of divisors on $C$. We 
say a divisor $D$ is \emph{defined over $k$} if $D=D^\sigma$ for all 
$\sigma\in G_k$. 

We put an equivalence relation on divisors: $D_1\sim D_2$ if 
$D_1-D_2=\divisor(f)$ for some rational function $f$ on $C$. 
So we need to define $\divisor(f)$. Let $k[C]=k[x,y]/f$ be the ring of regular 
functions on $f$, and let $k(C)$ be the field of fractions of $k[C]$; we call 
$k(C)$ the field of \emph{rational functions} on $C$. Any rational function 
$f$ induces a morphism $f:C\to \dP^1$. Since $C$ is projective, this map is 
either constant or surjective. We can now put 
\[
  \divisor(f) = \sum_{c\in C(\bar k)} \order_c(f) \cdot c.
\]
We put $\picard(C)=\divisors(C)/\divisor(k(C)^\times)$. 

Now we need to define the canonical divisor. It is the class of a nonvanishing 
differential on $C$. More formally, it's the divisor class corresponding to the 
line bundle $\Omega_C^1$. 

We put $l(D)=\dim \sL(D)$, where 
\[
  \sL(D) = \{f:C\epic \dP^1:\divisors(f)+D\geqslant 0\} \cup \{0\}.
\]
It is a theorem that $l(D) < \infty$ for all divisors $D$. 

Aside: let $[L:\dQ]<\infty$, i.e.~let $L$ be a number field. Then there is an 
exact sequence 
\[\begin{tikzcd}
  0 \ar[r] 
    & \cO_L^\times \ar[r] 
    & L^\times \ar[r] 
    & I_L \ar[r] 
    & \picard(O_L) \ar[r] 
    & 0 ,
\end{tikzcd}\]
where $I_L$ is the group of fractional ideals in $L$. See 
\cite{lorenzini-1996} for more on the parallel between 
algebraic number theory and algebraic geometry. This is analogous to the 
sequence 
\[\begin{tikzcd}
  0 \ar[r] 
    & k^\times \ar[r] 
    & k(C)^\times \ar[r, "\divisor"] 
    & \divisors^\circ(C) \ar[r] 
    & \picard^\circ(C) \ar[r] 
    & 0 .
\end{tikzcd}\]
Here the degree $\deg(\sum n_c c) = \sum n_c$. It is a basic fact that for 
$f\in k(C)^\times$, we have $\deg(f)=0$. In fact, $\picard(X)$ makes sense 
for $X$ any scheme (e.g.~a curve or the spectrum of $\cO_L$). 

If $C_{/\dQ}$ is a smooth projective curve, then $C(\dC)$ will be a compact 
Riemann surface, hence a torus with $g$ holes. We call $g$ the \emph{genus} of 
$C$. Geometrically, $g=\dim \h^0(\Omega^1)$. 

\begin{theorem}\label{thm:basic-facts}
Let $C_{/k}$ be a smooth projective curve, $D$ a divisor on $C$. Then 
\begin{enumerate}
\item $\deg D<0\Rightarrow l(D)=0$. 
\item $l(D)<\infty$. 
\item $D\sim D'\Rightarrow \sL(D)\simeq \sL(D')$. 
\item $l(K_C)=g$ and $\deg(K_C)=2 g-2$. 
\item $\deg(D)>2 g-2\Rightarrow l(D)=\deg(D)-g+1$. 
\end{enumerate}
\end{theorem}
\begin{proof}Choose $D$ cleverly (i.e.~$0$ or $K_C$) in the Riemann-Roch 
Theorem, and use $l(0)=1$. 
\end{proof}





\subsection{Elliptic curves}

If we're analyzing the case $g=0$, diophantine properties were known to the 
Greeks. For $g\geqslant 2$, Faltings tells that $\#C(\dQ)<\infty$. The 
remaining case is $g=1$. 

\begin{definition}
Let $k$ be a field. An \emph{elliptic curve over $k$} is a smooth projective 
curve $E_{/k}$ together with a point $0\in E(k)$. 
\end{definition}

Let $E_{/k}$ be an elliptic curve, $e\in E(k)$. What is $l(e)$? From 
\autoref{thm:basic-facts}, we get: 
\begin{center}
\begin{tabular}{c|cl}
$n$ & $l(n\cdot e)$ & basic of $\sL(n\cdot e)$ \\ \hline
1 & 1 & $\{1\}$ \\
2 & 2 & $\{1,x\}$ \\
3 & 3 & $\{1,x,y\}$ \\
4 & 4 & $\{1,x,y,x^2\}$ \\
5 & 5 & $\{1,x,t,x^2,x y\}$ \\
6 & 6 & $\{1,x,y,x^2,x y, x^3\text{ or } y^2\}$ 
\end{tabular}
\end{center}
The last set could have seven elements, but $\sL(6\cdot e)$ is 
$6$-dimensional. It follows that there is a dependence relation 
\[
  \alpha_0 + \alpha_1 x + \alpha_2 y+ \alpha_3 x^2 + \alpha_4 x y+\alpha_5 x^3 + \alpha_6 y^2 = 0 .
\]
We claim that $(\alpha_4,\alpha_6)\ne (0,0)$. Basic consideration of the 
orders of poles in the remaining equation at $e$ shows this. We can assume that 
$\alpha_6\ne 0$. Say $\alpha_6=0$ (then $\alpha_4\ne 0$). Consider the change 
of variables 
\begin{align*}
  x &\mapsto x + cy \\
  y &\mapsto y. 
\end{align*}
The new coefficient of $y^2$ is $\alpha_3 c^2 + \alpha_4 c$, and we can choose 
$c$ so that this is $\ne 0$. There's a better proof that $\alpha_6\ne 0$. If 
$\alpha_6=0$, since $\{1,x,y,x^2,x y,x^3\}$ is a basis of $\sL(6\cdot e)$, all 
the $\alpha_i=0$, which we can assume is not the case. 

Divide through by $\alpha_6$ to get the following:
\[
  y^2 + y (\alpha_2+\alpha_4 x) + \left(\frac{\alpha_2+\alpha_4 x}{2}\right)^2 - \left(\frac{\alpha_2+\alpha_4 x}{2}\right)^2 + \alpha_3 x^3 + \alpha_1 x + \alpha_0 = 0 .
\]
Replace $y$ by $y+\frac{\alpha_2+\alpha_4 x}{2}$; we get: 
\[
  y^2 = \gamma_3 x^3 + \gamma_2 x^2 + \gamma_1 x + \gamma_2 .
\]
Replace $y$ by $\gamma_3^2 y$ and $x$ by $\gamma_3 x$; this gives 
$x^3$ and $y^2$ the same coefficient. Rescale and we get 
\[
  y^2 = x^3 + \delta_2 x^2 + \delta_1 x + \delta_0 .
\]
Replace $x$ by $x-\frac{\delta_2}{3}$, and we get an equation 
\[
  y^2 = x^3 + A x+B 
\]
with $A,B\in k$. 

Note we have used that $k$ has characteristic not $2$ or 
$3$. Completion of squares doesn't work in characteristic $2$, and the 
first step in solving cubics doesn't work in characteristic $3$. So in 
characteristic $3$, you can get $y^2=(\text{cubic})$, but for characteristic 
$2$, you can't really simplify the equation at all. In general, if $A$ is a 
$d$-dimensional abelian variety, it is noted in \cite{serre-tate-1968} that 
primes $p\leqslant 2 d+1$ can be especially nasty. 

We will need to reduce elliptic curves modulo $2$ and $3$. 
Given $E_{/\dQ}$, there is an integer $N$, called the \emph{conductor} of $E$, 
that measures the ``badness'' of the singularities in the reductions of $E$. 
For $p\geqslant 3$, we have $v_p(N)\leqslant 2$. For 
$p=3$, we just have $v_3(N)\leqslant 4$, and for $p=2$, we can have 
$v_2(N)=6$. 

We'll continue under the assumption that $k=\dQ$. Write our curve 
$E_{/\dQ}$ as 
\[
  y^2=x^3 + \frac{N_1}{D} x + \frac{N_2}{D} ,
\]
with $N_1,N_2,D\in \dZ$. Make the change of variables 
\begin{align*}
  x &\mapsto x/D^2 \\ 
  y &\mapsto y/D^3 .
\end{align*}
Then multiply through by $D^6$ and relabel. We get 
\[
  y^2=x^3+ A x+ B x ,
\]
with $A,B\in \dZ$. For any prime $p$, say 
$p^4\mid A$ and $p^6\mid B$. Make the change of variables
\begin{align*}
  x &\mapsto p^2 x \\ 
  y &\mapsto p^3 y .
\end{align*}
you can rescale again until one of these possibilities fail. We end up with 
a Weierstrass form $y^2=x^3+A x+B$ with $p^4\mid A\Rightarrow p^6\nmid B$. This 
is the \emph{minimal model} of $E$. 

Now we can study $E_{/\dQ}$ and write $E=E_{A,B}$, given by $y^2=x^3+A x+B$, 
for $A,B\in \dZ$ with $(p^4\nmid A\text{ or }p^6\nmid B)$. Recall that 
$E_{A,B}$ is singular exactly if 
\[
  \begin{pmatrix} 2 y & 3 x^2+A\end{pmatrix} = 0
\]
for some $(x,y)\in E(\overline\dQ)$. This can happen only if 
$27 B^2+4 A^3=0$. The set of such $(A,B)$ is thin. 

This procedure works over $\dZ[\frac 1 6]\subset \dZ$. 





\subsection{Group law on an elliptic curve}

Let $E_{/k}$ be an elliptic curve in strong Weierstrass form 
$y^2=x^3+A x+B$. We will give $E$ the structure of an abelian 
variety over $k$. We declare $P+Q+R=0$ if $P$, $Q$, and $R$ lie on the same 
line through $E$ in $\dP^2$. This determines a morphism 
$E\times E\to E$ defined over $k$. From the definition, it is clear that 
addition is commutative. We won't verify that the group law is associative. 
It can be directly, but this is very tedious. A more conceptual approach is to 
use the Picard variety. To summarize: 

\begin{theorem}
Let $E_{/k}$ be an elliptic curve given by $y^2=x^3+A x+B$. Then $E$ has the 
unique structure of an abelian variety such that $(0:1:0)$ is the identity 
element. 
\end{theorem}
\begin{proof}
Existence follows from the fact that the map $E\to \picard^0(E)$ is an 
isomorphism of varieties over $k$. Uniqueness follows from Proposition 1.13 of 
\cite{moonenABV}. 
\end{proof}

Let $P=(\alpha,\beta)$. What is $2 P=P+P$? We can implicitly differentiate 
$y^2=x^3+A x+B$ to get 
\[
  y' = \left.\frac{3 x^2+A}{2 y}\right|_{(\alpha,\beta)} = \frac{3\alpha^2+A}{2 \beta} .
\]
We can solve the equation 
\[
  \left(\beta+\frac{3\alpha^2+A}{2\beta}(x-\alpha)\right)^2 = x^3+A x+B ,
\]
to see that the third root is $\frac{(3\alpha^2+A)^2}{4\beta^2} - 2\alpha$. So 
\[
  x_{2 P} = \frac{(3\alpha^2+A)^2}{4(\alpha^3+A\alpha+B)} - 2\alpha ,
\]
and $y_{2 P}$ can be computed similarly. 

Given $Q=(\alpha,\beta)\in E(\overline K)$, let's find $P$ such that 
$2 P=Q$. Since $\# E[2](\overline K)=4$, we expect four such $P$. It comes down 
to solving the equation 
\[
  \frac{(3\alpha^2+A)^2}{4(\alpha^3+A\alpha+B)} - 2\alpha = r .
\]
The polynomial on the left has degree $4$, so there will indeed be four $P$. 

In general, given $Q\in E(\overline K)$, the equation $n P=Q$ has $n^2$ 
solutions, provided $n$ is invertible in $k$. Also, if $n$ is invertible in 
$k$, then $E[n]\simeq (\dZ/n)^2$, and $G_K=\galois(\overline K/K)$ acts on 
$E[n]$ by group automorphisms. This is really easy. To verify this, all we need 
to check is that $\sigma(P+Q)=\sigma(P)+\sigma(Q)$ for all $\sigma\in G_K$ and 
$P,Q\in E(\overline K)$. But $x_{P+Q}$ and $y_{P+Q}$ are rational functions in 
the coordinates of $P$ and $Q$, \emph{defined over $K$}. If 
$f\in K(t)$ is a rational function, then $\sigma(f(x))=f(\sigma(x))$ for all 
$x\in \overline K$, whence $\sigma(x_{P+Q})=x_{\sigma P+\sigma Q}$. 

Given $E_{/\dQ}$, set $a_p=p+1-\# E(\dF_p)$, where $E(\dF_p)$ is the 
$\dF_p$-points of the projective curve associated to $E$. Note that 
$p+1=\dP^1(\dF_p)$. Note that this only makes sense for $p\nmid \Delta$. 

\begin{theorem}[Hasse]
$|a_p|\leqslant 2\sqrt p$. 
\end{theorem}
Recall the $L$-function of $E$ is defined in terms of the $a_p$: 
\[
  L(E,p) = \text{(finite \# of terms)} \prod_{p\nmid \Delta} \left(1-\frac{a_p}{p^s} + \frac{p}{p^{2s}}\right)^{-1} .
\]
If $\Re s=\frac 3 2+\epsilon$, then 
\[
  \left|-\frac{a_p}{p^s} + \frac{p}{p^{2s}}\right| \leqslant \frac{2}{p^{1+\epsilon}}+\frac{1}{p^{2+2\epsilon}} \leqslant p^{-1-\epsilon/2} ,
\]
at least for $p\gg 0$. We know that $\prod (1-p^{-s})^{-1}$ converges for 
$\Re s>1$. Thus $L(E,s)$ is defined for $\Re s>3/2$. 

We know that $|a_p|\leqslant p+1$. This gives well-definedness of $L(E,s)$ for 
$\Re s>2$, and it isn't too hard to get $\Re s>3/2$. The following theorem was 
proved in the course of Wiles' proof of Fermat's Last Theorem. 

\begin{theorem}
$L(E,s)$ has a meromorphic continuation to $\dC$. 
\end{theorem}

Recall the weak Birch and Swinnerton-Dyer conjecture:

\begin{conjecture}
$\order_{s=1} L(E,s) = \rank E$. 
\end{conjecture}
Note that $\rank E$ is global data, while the $a_p$ only depend on the reduction 
$E_{/\dF_p}$. The following theorem was known as the Sato-Tate conjecture. 

\begin{theorem}[Taylor et.~al.]
Define $\theta_p$ by $\frac{a_p}{2\sqrt p} = \cos(\theta_p)$ and 
$\theta_p\in [-\pi/2,\pi/2)$. Then 
\[
  \lim_{X\to \infty}\frac{\#\{p\leqslant X:c\leqslant \theta_p\leqslant d\}}{\pi(X)} = \frac 2 \pi \int_c^d \sin^2(u)\, \mathrm{d} u .
\]
\end{theorem}
\begin{proof}
In \cite{serre-1968}, Serre outlined a strategy. It relied on 
$L(\symmetric^n E,s)$ being analytic at $1$ for infinitely many $n$. At the 
time, we didn't even know this for $n=1$! This requires a vast generalization 
of Wiles et.~al.~(who worked with $n=1$). 
\end{proof}

In some ways, the Sato-Tate conjecture is unsatisfying because it says that 
``all (non-CM) elliptic curves are the same.'' On the other hand, the Birch and 
Swinnerton-Dyer conjecture is about how local data of elliptic curves gives 
information about global data. 

\begin{conjecture}[Harris]
Let $E_1$, $E_2$ be non-isogenous (up to twist) elliptic curves over $\dQ$. 
Then the Sato-Tate distributions of $E_1$, $E_2$ are independent. 
\end{conjecture}

See \cite{harris-ST} for a careful statement and motivation. 
There is an obvious family to pairwise non-isogenous families 
$\{E_1,\cdots,E_n\}$, but there isn't a clear approach to prove this. 





\subsection{Kummer theory}

Let $K\subset L$ be a Galois field extension. There is a version of Galois 
theory for $G=\galois(L/K)$, but we need to give $G$ a topology. Let a basis 
of $1$ be the subgroups of the form $G_x=\stabilizer_G(x)$ for $x\in L$. Under 
this topology (the \emph{Krull topology}), the group $G$ is compact, Hausdorff, 
and totally disconnected. See Chapter IV of \cite{neukirch-1999} for details. 

Let $k$ be a field, $p$ a prime invertible in $k$ such that $\dmu_p\subset k$. 
Let $k^{(p)}$ be the composite of all $\dZ/p$-Galois extensions of $k$ in 
$\overline k$. 

\begin{theorem}
The pairing $\galois(k^{(p)}/k)\times k^\times/p \to \dmu_p$ given by 
\[
  \langle \sigma,x\rangle = \frac{\sigma(x^{1/p})}{x^{1/p}} 
\]
is a well-defined perfect pairing of topological groups. 
\end{theorem}
\begin{proof}
Say $\langle \sigma,x_0\rangle=1$ for all $\sigma\in \galois(k^{(p)}/k)$. Then 
$\sigma(x_0^{1/p})=x_0^{1/p}$ for all $\sigma$. This implies 
$x_0^{1/p}$ is fixed by $\galois(k^{(p)}/k)$, which implies 
$x_0^{1/p}\in k$, which implies $x_0\in (k^\times)^p$. We'll prove the other 
direction later. 
\end{proof}

See \cite[IV \S 3]{neukirch-1999} for a proof. The theorem, correctly 
rephrased, works with arbitrary $n$ invertible in $k$. 

As an example, for $p=2$ and $k=\dR$, we see that there is only one 
$\dZ/2$-extension of $\dR$. This can be used to show that $\dC$ has no solvable 
extensions. 

There is a purely geometric approach to Kummer theory. Let $X$ be a scheme 
on which $n$ is invertible. The \emph{Kummer exact sequence} is the sequence 
\[
  1 \to \dmu_n \to \Gm \xrightarrow n \Gm \to 1 
\]
of \'etale sheaves on $X$. The long exact sequence in sheaf cohomology gives 
us a short exact sequence: 
\[
  1 \to \sO(X)^\times/n \to \h^1(X,\dmu_n) \to \picard(X)[n] \to 1 .
\]
Since $\dmu_n\simeq \dZ/n$, we have $\h^1(X,\dmu_n)=\hom(\pi_1(X),\dZ/n)$. 
When $X=\spectrum(k)$, we have $\picard(X)=0$, so 
$k^\times/n\simeq \hom(G_k,\dZ/n)$. For arbitrary $X$, the Kummer exact 
sequence can be used to get a pretty good grasp of $\pi_1(X)^\mathrm{ab}$. 
See \cite[5.8.3]{szamuely-2009} for details. 

Kummer theory is used to study solvable extensions in classical algebraic 
number theory. More generally, it is used in class field theory to build the 
abelian extensions of a field. Finally, it is used in local and global duality 
theorems in the cohomology of $G_\dQ$, $G_{\dQ_l}$. 

The Weil pairing is a kind of ``geometric Kummer theory.'' It is a perfect 
pairing $E[m]\times E[m]\to \dmu_m$. This means that 
$E[m]\simeq \hom(E[m],\dmu_m)=E[m]^\vee$, the dual in the correct sense. So 
$E[m]$ is self-dual in some sense. Even better, 
$E[m]^\vee\simeq \h^1_\mathrm{et}(E,\dZ/m)$, and the Weil pairing is comes from 
the cup-product 
pairing in \'etale cohomology: 
\[
  \h^1(E,\dZ/m)\times \h^1(E,\dZ/m) \to \h^2(E,\dZ/m).
\]

\begin{example}
Let $p=2$ and $k=\dQ$. Then 
$\dQ^{(2)}=\dQ(\sqrt{-1},\sqrt{l}\text{ all prime }l)$. Note that 
\[
  \galois(\dQ^{(2)}/\dQ) \simeq \prod_{l,-1} \dZ/2 .
\]
On the other hand, $\dQ^\times/2 = \bigoplus_{l,-1} \dZ/2$. The point is that 
the Kummer pairing is between a discrete group and a (uncountable) compact 
group. 
\end{example}

In general, Galois groups are either finite or uncountable (because they're 
compact). 





\subsection{Weil pairing}

We start with some needed facts, which are proved using Riemann-Roch. Let 
$k$ be a field, $n\geqslant 2$ an integer invertible in $k$, and let 
$E_{/k}$ be an elliptic curve. Suppose $D=\sum n_x(x)$ is a divisor on $E$. 
\begin{enumerate}
\item
If $\deg(D)=0$, then $D=\divisor(f)$ for some rational function $f$ on $E$ if 
and only if $\sum [n_x] x=0$ as an element of $E(\overline k)$. 

\item
If $x\ne 0$ in $E$, then $(x)-(0)\ne \divisor(f)$ in $\picard^\circ(E)$. 
\end{enumerate}

Given $x,y\in E[n]$, we'll define $e_n(x,y)\in \dmu_n$ following 
\cite[III \S 8]{silverman-2009}. By fact 1 above, the divisor 
$D=n(y)-n(0)$ is $\divisor(f)$ for some function $f$ on $E$. Let $y'$ be such 
that $[n]y'=y$, and consider 
\[
  [n]^\ast D = \sum_{z\in E[n]} (y'+z)-(z) ,
\]
where $y'+z$ is defined using the addition in $E$. This is a degree-$0$ 
diivisor, and within $E$, we have $\sum_{z\in E[n]}(y'+z-z)=n^2 y' = n y = 0$. 
Thus $[n]^\ast D=\divisor(g)$ for some $g$, and it is easy to check that we 
must have $f = c g^n$ for some $c\in \overline k$. We can replace $g$ by a 
scalar multiple tog et $f=g^n$. Note that 
$g(t+x)^n = f([n] t+[n] x) = f([n] t) = g(t)^n$. Thus we can define 
\[
  e_n(x,y) = \frac{g(t+x)}{g(t)}
\]
for any $t$ in the domain of $g$. Note that this definition is 
asymmetric with respect to $x,y$. Also $f,g$ are independent of $y$. 

\begin{theorem}
Let $E_{/k}$ be an elliptic curve. Then: 
\begin{enumerate}
  \item $e_n(x_1+x_2,y) = e_n(x_1,y) e_n(x_2,y)$ 
  \item $e_n(x,y_1+y_2) = e_n(x,y_1) e_n(x,y_2)$ 
  \item $e_n(x,x)=1$, so $e_n(x,y) = e_n(y,x)^{-1}$ 
  \item $e_n(x,y)^\sigma = e_n(x^\sigma,y^\sigma)$ for all $\sigma\in G_k$. 
  \item $e_n(x,y)=1$ for all $x$ implies $y=0$, and vice versa. 
\end{enumerate}
\end{theorem}
\begin{proof}
This is Proposition III.8.1 in \cite{silverman-2009}. 

1. We have 
\begin{align*}
  e_n(x_1+x_2,y) 
    &= \frac{g(t+x_1+x_2)}{g(t)} \\
    &= \frac{g((t+x_1)+x_2)}{g(t+x_1)} \frac{g(t+x_1)}{g(t)} \\
    &= e_n(x_1,y) e_n(x_2,y) .
\end{align*}
The other part is more involved. Call $y_3=y_1+y_2$. Let 
$f_i,g_i$ be the functions for $y_i$. We must relate $g_3$ to $g_1,g_2$. 
Put $D=(y_3)-(y_1)-(y_2)+(0)$; this has degree $0$, and the sum of its points 
is $0$ in $E$, so $D=\divisor(h)$ for some $h$. Recall that 
$\divisor(f_i) = n(y_i) - n(0)$. Thus 
\[
  \divisor\left(\frac{f_3}{f_1 f_2}\right) = \divisor(h^n) .
\]
Thus $f_3=c f_1 f_2 h^n$ for some $c$. It follows that 
$f_3\circ [n] = c f_1\circ [n] f_2\circ [n] (h\circ [n])^n$, so 
$g_3^n = g_1^n g_2^n (h\circ [n])^n$. Take $n$-th roots: we get 
$g_3 = \widetilde c g_1 g_2 (h\circ [n])$. We can now compute: 
\begin{align*}
  e_n(x,t_3) 
    &= \frac{g_3(t+x)}{g_3(t)} \\
    &= \frac{g_1(t+x)g_2(t+x)h([n] t+[n] x)}{g_1(t) g_2(t) h([n] t)} \\
    &= e_n(x,y_1) e_n(x,y_2) .
\end{align*}

4. We compute 
\[
  e_n(x,y)^\sigma 
    = \left(\frac{g(t+x)}{g(t)}\right)^\sigma 
    = \frac{g^\sigma(t^\sigma+x^\sigma)}{g^\sigma(t^\sigma)} 
    = e_n(x^\sigma,y^\sigma) .
\]

We leave the rest of the proof as an exercise. 
\end{proof}

The alternating property and Galois-equivariance of the Kummer pairing have a 
nice consequence. Choose a generating set $x,y\in E[n]$, $\sigma\in G_k$. Then 
$\sigma$ acts on $E[n]$ via some matrix 
\[
  \begin{pmatrix} a & b \\ c & d \end{pmatrix}\in \GL_2(\dZ/n)
\]
with respect to the basis $\{x,y\}$. Suppose $e_n(x,y)=\zeta\in \dmu_n$. Then 
$\sigma(\zeta) = \zeta^{\chi(\sigma)}$, where $\chi:G_k\to \GL_1(\dZ/n)$ is 
the cyclotomic character. Write $\rho_{E,n}:G_k\to \GL_2(\dZ/n)$ for the 
representation coming from $E$. We claim that $\det\rho_{E,n} = \chi_n$. All 
we need to show is that $e_n(x,y)^\sigma = \zeta^\sigma = \zeta^{a d-b c}$. We 
know that $e_n(x,y)^\sigma = e_n(x^\sigma,y^\sigma) = e_n(a x+c y, b x+d t)$; 
we're left with $e_n(x,y)^{a b-c d}$, as desired. 





\subsection{Isogenies}

An \emph{isogeny} will be morphisms between elliptic curves, defined over the 
base field. Since elliptic curves are varieties, we should require these 
morphisms to be polynomial (or rational). Since elliptic curves have a group 
law, we also want these morphisms to respect this structure. Summing up: if we 
think of elliptic curves as group objects in the category of projective 
varieties (projective $\Rightarrow$ abelian), an isogeny is a non-constant 
morphism of such objects. Isogenies always have finite kernel. 

If $\phi:E_1\to E_2$ is defined over $k$, then $\ker\phi$ is $G_k$-stable as a 
group (not pointwise). Given $E_{/\dQ}$ and an isogeny $\phi:E\to E$, usually 
$\phi$ is $[n]$ for some $n\in\dZ\smallsetminus 0$. When there are isogenies 
not of this form, amazing congruences happen. 

\begin{example}
Let $\Delta=q \prod_{n\geqslant 1} (q-1^n)^{24} = \sum \tau(n) q^n$; the first 
few terms are: 
\[
  \Delta = q-24 q^2 + 252 q^3 - 1472 q^4 + 4830 q^5 - 6048 q^6 + \cdots .
\]
Ramanujan observed that $(m,n)=1$ implies $\tau(m n) = \tau(m) \tau(n)$. He 
also observed that for a prime $p$, 
$\tau(p^{r+1}) = \tau(p) \tau(p^r) - p^{11} \tau(p^{r-1})$. Also, 
$|\tau(p)|\leqslant 2 p^{11/2}$ (this is hard: it was proved by Deligne in the 
course of the Weil conjectures). Finally, 
\[
  \tau(n) \equiv \sigma_{11}(n) = \sum_{d\mid n} d^{11} \pmod{691} .
\]
This congruence is a manifestation of the ``existence of a non-trivial 
isogeny.'' 
\end{example}

This example is one of the first cases of a congruence between a \emph{cusp 
form} and \emph{Eisenstein series}. The first paper to make much use of this 
idea is \cite{ribet-1976}, in which Ribet used modular forms to prove the 
converse of Kummer's criterion. The ideas here were crucial in the proof of 
Iwasawa's Main Conjecture, Skinner and Urban's recent work, and many other 
parts of modern number theory. See Mazur's survey article 
\cite{mazur-2011} for an excellent overview of this circle of ideas. 




