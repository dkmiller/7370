% !TEX root = 7370.tex

\section{Introduction}





\subsection{Disclaimer}

These notes originated in the course MATH 7370: Ranks of elliptic curves, 
taught by Ravi Ramakrishna at Cornell University. The notes are not necessarily 
an exact reflection of the material as it was covered in class. 





\subsection{Motivation}

Our main references will be the preprints 
\cite{
  bhargava-shankar-1,
  bhargava-shankar-2,
  bhargava-shankar-3,
  bhargava-shankar-4,
  bhargava-skinner,
  bhargava-skinner-zhang}. 

The fundamental question is how to solve general diophantine equations 
\[
  f_1(x_1,\dots,x_m) = \cdots = f_n(x_1,\dots,x_m) = 0
\]
for the $f_i$ polynomials over $\dQ$ or $\dZ$ and the $x_i$ in $\dQ$ or $\dZ$. 
This is very hard, so we'll focus on curves. This comes down to solving 
equations $f(x,y)=0$ where $f\in \dZ[x,y]$ or $\dQ[x,y]$. Or we could 
homogenize and consider points on the associated projective curve. 

\begin{example}
Consider the equation $x^2+y^2=1$. There is one obvious solution $(0,1)$. From 
this we get all the other solutions by looking at lines $y=m x+1$ with 
$m\in \dQ$. The equation $x^2+(m x+1)^2=1$ already has one rational solution, 
so the other must be rational, and all rational solutions to $x^2+y^2=1$ are of 
this form. So solutions to $x^2+y^2=1$ are in bijection with $\dQ$. More 
geometrically, any nice (that is, smooth, proper and geometrically integral) 
curve $C_{/\dQ}$ that has one rational point is isomorphic (over $\dQ$) to 
$\dP^1_{/\dQ}$. 
\end{example}

Any smooth projective curve $C_{/\dQ}$ is the homogeneous zero-set of an 
equation $f(x,y,z)=0$. The set $C(\dC)$ is naturally a compact Riemann surface 
of some genus $g$. 

\begin{theorem}[Faltings]
If $C_{/\dQ}$ be a curve of genus $g>1$. Then $\# C(\dQ)<\infty$. 
\end{theorem}
There still isn't an effective algorithm for curves of genus $\geqslant 2$ that 
will produce the set of zeros of $f$. 

We've seen that when $g=0$, either $C(\dQ)=\varnothing$ or 
$C\simeq \dP^1$. When $g\geqslant 2$, Faltings' theorem tells us that 
$\# C(\dQ)<\infty$. Our concern is the remaining case $g=1$. If $C_{/\dQ}$ is 
a genus one curve and $C(\dQ)=\varnothing$, there isn't much to do. For the 
remainder, we will concentrate on nice curves $E_{/\dQ}$ together with a chosen 
point $0\in E(\dQ)$. Such curves are called \emph{elliptic curves}. 

Let $E_{/\dQ}$ be an elliptic curve. Basic algebraic geometry involving little 
more than Riemann-Roch shows that $E$ can be written in the form 
$y^2=x^3+A x+B$ with $A,B\in \dZ$. Projectively, this is
$y^2 z=x^3 + A x z^2+B z^3$. The point $(0:1:0)$ is the $0\in E(\dQ)$. 

\begin{example}
Consider the curve $y^2=x^3-9 x+9$. There are obvious solutions 
$(1,1)$, $(3,3)$. The line through them is $y=x$. Since the cubic 
$x^2=x^3-9 x-9$ already has two rational solutions, the third must be rational. 
This third point is  $(-3,-3)$. In general, if $P,Q,R\in E$, we say that 
$P+Q+R=0$ if $P,Q,R$ are the intersection points of a line and $E$ in 
$\dP^2$. For example, one can check that 
\begin{align*}
  (-3,3)+(1,1) &= (\frac 9 4, -\frac 3 8) \\
  (-3,3)+\left(\frac 9 4, -\frac 3 8\right) &= \left(\frac{57}{49},-\frac{111}{343}\right).
\end{align*} 
\end{example}

Any elliptic curve $E_{/\dQ}$ has the canonical structure of a group variety 
over $\dQ$. For our purposes, this means that there are regular maps 
$m:E\times E\to E$ and $i:E\to E$ that make $E(\dC)$ with the maps induced by 
$m,i$ an honest group with identity element $0$. But for any field 
$F\supset \dQ$, there is an abelian group $E(F)$ which is functorial in $F$. 

\begin{theorem}[Mordell-Weil]
Let $E_{/\dQ}$ be an elliptic curve. Then $E(\dQ)$ is finitely generated. 
\end{theorem}

Thus it makes sense to define the \emph{rank} of $E$ to be 
$\rank(E)=\rank_\dZ E(\dQ)$. Given a specific elliptic curve $E$, we might ask: 
what are its rank and torsion? The second part is easy. In his paper 
\cite{mazur-1977}, Mazur proved that the torsion part of $E(\dQ)$ is one of the 
following groups:
\begin{center}
\begin{tabular}{rl}
$\dZ/m$ & for $m\leqslant 10$ or $m=12$ \\
$(\dZ/2)\oplus (\dZ/2 n)$ & for $n\leqslant 4$ 
\end{tabular}
\end{center}
This paper is very beautiful, and inspired a lot of amazing mathematics. If $E$ 
is a specific elliptic curve, it is easy to explicitly check which of these the 
torsion subgroup of $E(\dQ)$ is. 





\subsection{\texorpdfstring{$L$}{L}-functions of elliptic curves}

For $p$ sufficiently large, $E_{/\dF_p}$ is an elliptic curve. Write 
$\# E(\dF_p)=p+1-a_p$; we have the \emph{Hasse bound} $|a_p|\leqslant 2\sqrt p$. 

We define the $L$-function of $E$ as 
\[
  L(E) = \prod_{p \text{ bad}} ? \times \prod_{p\text{ good}} \left(1-\frac{a_p}{p^s} + \frac{1}{p^{2s-1}}\right)^{-1} .
\]
There only finitely many bad primes (they are the ones dividing 
the discriminant $\Delta=-(4 A^3+27 B^2)$ if $E$ is the curve 
$y^2=x^3+A x+B$.), and the factors at the bad primes are rational functions in 
$p^s$. Nonetheless, this is the world's most awful definition! Let's give a 
better one. We know that 
$E[l^n]\simeq (\dZ/l^n)^2$. This group admits an action of 
$G_\dQ=\galois(\bar\dQ/\dQ)$, an uncountable group with a natural 
compact, totally disconnected topology. We can paste these actions together 
to get a representation $\rho:G_\dQ\to \GL_2(\dZ_l)$. There are conjugacy 
classes $\frobenius_p\in G_\dQ$, and $p\nmid \Delta$, we have 
\[
  \rho(\frobenius_p) \sim \begin{pmatrix} \alpha_p \\ & \beta_p \end{pmatrix} .
\]
It turns out that 
\[
  1-\frac{a_p}{p^s} + \frac{1}{p^{2s-1}} = \left(1-\frac{\alpha_p}{p^s}\right)\left(1 - \frac{\beta_p}{p^s}\right) .
\]
The function $L(E,s)$ may not seem that complicated, but note that the zeta 
function built from the trivial representation $G_\dQ\to \GL_1(\dC)$ is the 
Riemann zeta function 
\[
  \zeta(s)=\prod_p \left(1-\frac{1}{p^s}\right)^{-1} .
\]
So we should expect the behavior of $L(E,s)$ to be very subtle! Most zeta 
functions encountered in number theory come from Galois representations in this 
manner. 

\begin{conjecture}[Birch, Swinnerton-Dyer]
Let $E_{/\dQ}$ be an elliptic curve. Then $\rank(E)=\order_{s=1} L(E,s)$. 
\end{conjecture}

In fact, Birch and Swinnerton-Dyer predicted the leading term at $s=1$ of 
$L(E,s)$ in terms of arithmetic data attached to $E$. The conjecture implicitly 
includes the assertion that $L(E,s)$ has a meromorphic continuation past $s=1$. 
By the work of Wiles and his followers, we know that $E$ is modular, hence its 
$L$-function agrees with that of a modular form. Hecke showed that modular 
$L$-functions have such meromorphic continuations, so no problems there. 

It is known that $L(E,s)$ satisfies a functional equation 
\[
  L(E,s) = \pm ? L(E,2-s) 
\]
where $?$ is easily computable. If the sign is negative, we know that 
$L(E,1)=0$, which suggests that $\rank(E)>0$.

\begin{conjecture}[Goldfeld, Katz-Sarnak]
Let $E_{/\dQ}$ be an elliptic curve. Then 
\[
  \rank(E) = \begin{cases} 0 & \text{50\% of the time} \\ 1 & \text{50\% of the time.} \end{cases}
\]
\end{conjecture}

We have to be careful about what we mean by ``50\% of the time'' as there are 
infinitely many elliptic curves over $\dQ$. To make probabilistic statements 
precise, we'll order elliptic curves. 





\subsection{Asymptotics of rank}

Given $A,B\in \dZ$ with $4 A^3+27 B^2\ne 0$, write $E_{A,B}$ for the elliptic 
curve $y^2=x^3+A x+B$. Define its \emph{height} to be 
$\height(E)=\max(4 |A|^3,27 |B|^2)$. It's easy to check that asymptotically, 
about $X^{5/6}$ curves have height $\leqslant X$. Let $A(X)$ be the average 
rank of the set of elliptic curves with height $\leqslant X$. Assuming the 
Generalized Riemann  Hypothesis, these were the bounds on $A(X)$ prior to 
Bhargava's work: 
\begin{align*}
  \limsup_{X\to \infty}  A(X) \leqslant 
    & 2.3 && \text{\cite{brumer-1992}} \\
    & 2.0 && \text{\cite{heath-brown-2004}} \\
    & 1.79 && \text{\cite{young-2006}}
\end{align*}
Unconditionally, all that was known was the trivial bounds
\[
  0\leqslant \liminf_{X\to \infty} A(X) \leqslant \limsup_{X\to \infty} A(X) \leqslant \infty .
\]
Now, after Bhargava's work, we have
\[
  0.2 \leqslant \liminf_{X\to \infty} A(X) \leqslant \limsup_{X\to \infty} A(X) \leqslant 0.885 .
\]
We also know that $\order_{s=1} L(E,s) = \rank E(\dQ)$ at least 
$66.43\%$ of the time, and there is a strategy to get $100\%$. Unfortunately, 
this strategy only says things about the (conjecturally $100\%$ of) elliptic 
curves with $\rank\leqslant 1$. 

There is a folklore question: is $\rank(E)$ bounded? About 15 years ago, most 
people thought this wasn't bounded, now, most people think it is bounded. 





\subsection{Proof strategy}

For any abelian group $A$, we it is trivial that
$\rank(A)\leqslant \dim_{\dF_p}(A/p)$. We can prove that the set of elliptic 
curves with $\dZ/p\subset E[p]$ is density zero, so it won't affect density 
arguments. 

There is an exact sequence 
\[\begin{tikzcd}
  0 \ar[r] 
    & E[p](\overline\dQ) \ar[r] 
    & E(\bar\dQ) \ar[r, "p"]
    & E(\bar\dQ) \ar[r] 
    & 0 .
\end{tikzcd}\]
Take $G_\dQ$-invariants and pass to the long exact sequence in Galois 
cohomology: 
\[\begin{tikzcd}
  0 \ar[r] 
    & E(\dQ)[p] \ar[r] 
    & E(\dQ) \ar[r, "p"] 
    & E(\dQ) \ar[r] 
    & \h^1(G_\dQ,E[p]) \ar[r] 
    & \cdots
\end{tikzcd}\]
Some fiddling around involving completions of $\dQ$ gives a short exact sequence 
\[\begin{tikzcd}
  0 \ar[r] 
    & E(\dQ)/p \ar[r] 
    & \selmer_p(E) \ar[r] 
    & \sha(E)[p] \ar[r]
    & 0 .
\end{tikzcd}\]
In the 60s, Cassels discovered a nice geometric way of describing $\selmer_p E$. 
Elements correspond to some geometric objects (a map $C\to E$), which in turn 
correspond to a embedding $C\hookrightarrow \dP^{p-1}$ (at least if 
$p\geqslant 3$). When $p=3$, $C$ is the zero locus of a ternary cubic form. 

Ternary cubics have two invariants $I,J$. These correspond to $A$ and $B$. So 
rather than counting elliptic curves, we can count ternary cubics (up to 
equivalence). 

The quadratic forms $x^2+y^2$ and $(u+v)^2+v^2$ are isomorphic over $\dZ$. 
General quadratic forms $a x^2+b x y+c y^2$ over $\dZ$ have an invariant 
$b^2-4 a c$ (invariant for action of $\SL_2(\dZ)$). Essentially, you have 
to find a fundamental domain for the action of $\SL_2(\dZ)$ on the upper 
half plane. It's a lot harder to find a fundamental domain for the action 
of $\SL_3(\dZ)$ on a bigger space. 

For $p\leqslant 5$, you count equivalences of forms; this is done by 
finding a fundamental domain for the action of some arithmetic group. 
This has been done for $p=2,3,5$. 

Essentially, one counts lattice points in $G\backslash V_\dZ$. The problem 
is, the fundamental domains can have cusps. It turns out that the 
nonzero elements of $\selmer_p E$ appear in the ``main body'' of the 
fundamental domain, and only $0$ appears in the cusps. 

\begin{theorem}[Bhargava-Shankar]
The average size of $\selmer_p E=p+1$ for $p=2,3,5$. 
\end{theorem}

Put $x=\dim_{\dF_p} E(\dQ)/p$. It is easy to check that 
$(p^2-p) x+2 p-p^2 \leqslant p^x$. Note that $p^x$ is $p+1$ on average. 
It follows that on average, $\rank E\leqslant 1+\frac{1}{p(p-1)}$. 
If we could prove this for all $p$, we get average $\leqslant 1$. Bhargava 
and Shankar use work of the Dokchitser's to get bounds $<1$ without proving 
that the average of $\#\selmer_p E=p+1$ for all $p$. 





\subsection{Review of the overview}

\emph{Note}: the above introduction was given in a department-wide lecture. 
What follows is a general overview given in the first day the class met. 

Let $E_{/\dQ}$ be an elliptic curve. This is a smooth genus $1$ curve over 
$\dQ$ of the form $y^2=x^3+A x+B$, where $A,B\in \dZ$, such that 
\[
  p^4\mid A\Rightarrow p^6\nmid B .
\]
It is a basic fact that every isomorphism class of elliptic curves over 
$\dQ$ has a unique representative of this form. 

Throughout, we'll write an equation $y^2=x^3+A x+B$ to mean the subvariety of 
$\dP^2$ cut out by the homogenization $y^2 z=x^3+A x z^2+B z^3$ of this 
equation. So $E(\dQ)$ consists of solutions to $y^2=x^3+A x+B$, as well as the 
point $(0:1:0)$ ``at infinity.'' Some useful facts:
\begin{enumerate}
\item
$E(\dQ)$ is a finitely-generated abelian group (Mordell-Weil). 

\item
$E(\dQ)_\mathrm{tors}$ is ``understood'' All possible such 
subgroups have been written down (see Mazur's list above), and they all 
occur. There is an effective algorithm to determine the torsion part of the 
Mordell-Weil group of an elliptic curve $E$. Since the group law is given by 
polynomials, the ``multiply by $n$'' map $[n]:E\to E$ is a polynomial. 
Simply check whether the roots of $[n]$ lie in $\dQ$. There are only finitely 
many possible $n$, so the algorithm will terminate. 

\item
$\#E(\dQ)_\mathrm{tors}=1$ one hundred percent of the time. 
\end{enumerate}

Recall that the \emph{height} of $E_{A,B}:y^2 z=x^3+A x z^2+b z^3$ is 
$\max(4|A|^3,27 B^2)$. This definition is actually 
pretty natural if you know the definition of the discriminant of $E$. 

Recall that our goal is to study the average rank of $E_{/\dQ}$ with height 
$\leqslant X$, as $X\to \infty$. That is, we are interested in the asymptotics 
of 
\[
  \lim_{X\to \infty} \frac{\sum_{\height E\leqslant X} \rank(E)}{\#\{E:\height E\leqslant X\}} .
\]
Currently it is not known if this limit exists, but conjecturally it is $1/2$. 
Note that $\#\{E:\height E\leqslant X\}=O(X^{1/3}\cdot X^{1/2})=O(X^{5/6})$. We 
have to be careful, because some of the curves $E_{A,B}$ will be singular, but 
this happens on a thin set, so it will not affect our calculations. 

Note that for any finitely-generated abelian group $A$, we have 
\[
  \rank(A) = \dim_\dQ(A\otimes\dQ) \leqslant \dim_{\dF_p}(A\otimes\dF_p) .
\]
So to get upper bounds on $\rank(E)$, it suffices to bound 
$\dim_{\dF_p}(E(\dQ)/p)$. 

\begin{theorem}
Let $E_{/\dQ}$ be an elliptic curve. Then $E(\overline\dQ)$ is a divisible 
group. 
\end{theorem}
\begin{proof}
Given $n\in \dZ$ and $x\in E(\overline\dQ)$, we need some $y\in E(\overline\dQ)$ 
such that $n\cdot y=x$. The coefficients of the polynomial $[n]$ will be 
algebraic over $\dQ$, so all solutions to $[n] y=x$ will be algebraic over $\dQ$. 
\end{proof}

By \cite[Cor 5.11]{moonenABV}, we have $E[n](\overline\dQ)\simeq (\dZ/n)^2$ for 
all $n\geqslant 1$. Thus $E(\overline\dQ)_\mathrm{tors}\simeq (\dQ/\dZ)^2$. 
Since $E(\overline\dQ)$ is divisible, we have an exact sequence 
\begin{equation}\label{eq:ell-ses}
\begin{tikzcd}
  0 \ar[r] 
    & E(\overline\dQ)[p] \ar[r] 
    & E(\overline\dQ) \ar[r, "p"] 
    & E(\overline\dQ) \ar[r] 
    & 0 .
\end{tikzcd}
\end{equation}
Recall that $E(\dC)\simeq S^1\times S^1$ as real Lie groups. The finite group 
$E(\overline\dQ)[p]\simeq (\dZ/p)^2$ comes with a standard pairing, the 
\emph{Weil pairing}, coming from the cup product on $\h^1(\dQ,E[p])$. 

Let $\overline\dQ$ be an algebraic closure of $\dQ$, and let 
$G_\dQ=\galois(\overline\dQ/\dQ)$. This is an uncountable, totally disconnected 
compact topological group, and its cohomology has a good duality theory; see for 
example \cite{neukirch-schmidt-winberg-2008}. There is an analogy between 
$G_\dQ$ and 3-manifolds, see e.g.~\cite{morishita-2012}. 

The group $G_\dQ$ acts on $E(\overline\dQ)$, respecting the group structure. So 
we can take its cohomology. 
The functor $M\mapsto M^{G_\dQ}$ is left exact, so we can do the usual nonsense 
with enough injectives and derived functors to get $\h^\bullet(G_\dQ,M)$. Or we 
can write a direct definition using cocycles and coboundaries. Often, to save 
space, if $F$ is a field, $G_F$ its absolute Galois group and $M$ a 
$G_F$-module, we'll write $\h^\bullet(F,M)$ instead of $\h^\bullet(G_F,M)$. 

A commutative diagram 
\[\begin{tikzcd}
  \overline\dQ \ar[r, hook] 
    & \dC \\
  \dQ \ar[r, hook] \ar[u, hook] 
    & \dR \ar[u, hook] 
\end{tikzcd}\]
gives rise to an injection $G_\dR=\galois(\dC/\dR)\monic G_\dQ$. Also, for 
each prime $l$ we get an inclusion $G_{\dQ_l}\monic G_\dQ$. The groups 
$G_{\dQ_l}$ are much bigger than $G_\dR=\dZ/2$, but they're pretty well-behaved 
[for example, they're pro-solvable in a nice way]. The group 
$G_\dQ$ is very poorly understood. We know its abelianization quite well, and 
2-dimensional representations reasonably well, but higher-dimensional 
representations not well at all. 

The long exact sequence in cohomology coming from \eqref{eq:ell-ses} gives 
rise to a natural short exact sequence 
\[\begin{tikzcd}
  0 \ar[r] 
    & E(\dQ)/p \ar[r] 
    & \h^1(G_\dQ,E[p]) \ar[r] 
    & \h^1(G_\dQ,E)[p] \ar[r] 
    & 0
\end{tikzcd}\]
Unfortunately, $\h^1(G_\dQ,E[p])$ is infinite-dimensional, so this doesn't seem 
very helpful. But we can extend this to a diagram 
\[\begin{tikzcd}
  0 \ar[r] 
    & E(\dQ)/p \ar[r] \ar[d] 
    & \h^1(G_\dQ,E[p]) \ar[r] \ar[d] 
    & \h^1(G_\dQ,E)[p] \ar[r] \ar[d] 
    & 0 \\
  0 \ar[r] 
    & \prod_v E(\dQ_v)/p \ar[r] 
    & \prod_v \h^1(G_{\dQ_v},E[p]) \ar[r] 
    & \prod_v \h^1(G_{\dQ_v},E)[p] \ar[r] 
    & 0 .
\end{tikzcd}\]
Here, by convention $v$ ranges over all primes \emph{and $\infty$}, and we 
put $\dQ_\infty=\dR$. The vertical maps come from basic functoriality 
(restriction) of group cohomology. Put 
\[
  \selmer_p(E) = \ker\left(\h^1(G_\dQ,E[p]) \to \prod_v \h^1(G_{\dQ_v},E)[p]\right) .
\]
This is finite-dimensional, and $E(\dQ)/p\monic \selmer_p E$. The group 
$\selmer_p(E)$ is measuring some local-global stuff. That is, cohomology 
classes in $\h^1(G_\dQ,E[p])$ that ``are'' trivial locally everywhere. 

In \cite{cassels-1962}, Cassels showed that elements of $\selmer_p(E)$ are in 
bijection with locally soluble $p$-coverings of $E$. 

Briefly, the map $p:E\to E$ has field of definition $\dQ$. Consider the varieties 
$U:x^2+y^2=1$ and $V:x^2+y^2=3$. There is an isomorphism 
$U_{\dQ(\sqrt 3)} \to V_{\dQ(\sqrt 3)}$ by $(x,y)\mapsto (\sqrt 3x,\sqrt 3 y)$, 
but this isomorphism isn't defined over $\dQ$. A \emph{locally soluble 
$p$-covering} of $E$ is an isomorphism $\phi:C\iso E$ (not necessarily 
defined over $\dQ$) such that $[p]\circ \phi$ is defined over $\dQ$. Moreover, 
$C$ must have points over $\dR$ and all $\dQ_l$. 

Locally soluble $p$-coverings of $E$ give degree-$p$ divisors on $C$. This 
gives a map $C\to \dP^{p-1}$, which is an embedding of degree $3$ if 
$p\geqslant 3$. 

Say $p=3$. A degree-$3$ curve in $\dP^2$ is easy to describe. So elements in 
$\selmer_3(E)$ correspond to degree-$3$ cubics over $\dQ$ (up to equivalence). 

Let's switch gears. Consider binary quadratic forms over $\dZ$. These are just 
polynomials $a x^2+b x y+c y^2$ for $a,b\in \dZ$. These have a natural 
$\SL_2(\dZ)$-action coming from 
\[
  \begin{pmatrix} x \\ y \end{pmatrix} \mapsto \begin{pmatrix} a & b \\ c & d \end{pmatrix} = \begin{pmatrix} a x+b y \\ c x+d y \end{pmatrix} .
\]
But $\SL_2(\dZ)$ has generators $\begin{pmatrix} 1 & 1 \\ & 1 \end{pmatrix}$ and 
$\begin{pmatrix} & -1 \\ 1 \end{pmatrix}$. 

Claim: $d=b^2-4 a c$ is an invariant of the action of $\SL_2(\dZ)$. We'll only 
check $\begin{pmatrix} 1 & 1 \\ & 1 \end{pmatrix}$-invariance. We get 
\begin{align*}
  a(x+y)^2+b (x+y)y+c y^2 
    &= a x^2 + (2 a+b)x y + (c+b+a) y^2 ,
\end{align*}
and simply check that $(2 a+b)^2 - 4 a(a+b+c)=b^2-4 a c$. 

\begin{theorem}[Gauss]
Given $d$, there exist only finitely many inequivalent binary quadratic forms 
over $\dZ$ with discriminant $d$. 
\end{theorem}

It's natural to ask: ``how many are there?'' We'll ask the exact same question 
for ternary forms. In that case there are two invariants $I,J$. Proving that these 
are invariant is a simple computation. What is harder is the analogue of 
\[
  \dZ[b^2-4 a c]= \dZ[a,b,c]^{\SL_2(\dZ)} .
\]

A general theorem of Borel and Harish-Chandra says that there 
are still only finitely many equivalence classes of forms with any given 
pair of invariants. 

Q. what values $d$ occur? [easy: $d\equiv 0$ or $1\pmod 4$]

Q. What is the arithmetic significance of 
\[
  h(d)=\#\{\text{inequivalent quadratic forms with discriminant }d\}
\]
It turns out that $h(d)$ is the \emph{narrow class number} of $\dQ(\sqrt d)$. 
This measures failure of unique factorization in that field. 

Q. What is the average value of $h(d)$?

\begin{theorem}[Mertens, Siegel]
(a) For $-X<d<0$, the average value of $h(d)$ is $\frac{\pi}{18} X^{3/2} + O(X^{3/2-\varepsilon})$ for some explicit $\varepsilon>0$. 

(b) For $0<d<X$, the average value of $h(d)\log(\epsilon_d)$ is 
$\frac{\pi^2}{18} X^{3/2}+O(X^{3/2-\varepsilon})$. Here 
$\epsilon_d$ is a fundamental unit of $\dQ(\sqrt d)$. 
\end{theorem}

What's different about $p=2$ and $p>2$? When $p=2$, 
$C\to \dP^{p-1}$ isn't an embedding. But it is a degree-2 cover with four 
branch points. This comes down to counting binary quartic forms up to 
equivalence. Let 
\[
  a x^4 + b x^3 y + c x^2 y^2 + d x y^3 + e y^4 
\]
be such a form. The group $\SL_2(\dZ)$ acts as before. If 
\begin{align*}
  I &= 12 a e - 3 b d + c^2 \\
  J &= 72 a c e + 9 b c d - 27 a d^2 - 27 e b^2 - 2 c^3 .
\end{align*}
then we have 

\begin{theorem}
$\dZ[I,J] = \dZ[a,b,c,d,e]^{\SL_2(\dZ)}$. 
\end{theorem}
\begin{proof}
Find source [not Hilbert!]
\end{proof}

So counting elements of $\selmer_2(E)$ of $E$ of height $X$ corresponds to counting 
binary quartic forms via $(I,J)\leftrightarrow (A,B)$. 

\begin{theorem}
The number of such forms up to height $X$ with 
\begin{align*}
  4\text{ real roots} && \frac{4}{\zeta(2)}{135} X^{5/6}+O(X^{3/4}) \\
  2\text{ real roots} && 32 \frac{\zeta(2)}{135} X^{5/6} + O(X^{3/4}) \\
  \text{no real roots} && 8 \frac{\zeta(2)}{135} X^{5/6} + O(X^{3/4}) .
\end{align*}
\end{theorem}

\begin{corollary}
When ordered by height, $\average(\#\selmer_2)=2+1$. 
\end{corollary}

